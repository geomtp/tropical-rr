\documentclass[a4paper,dvipdfmx,reqno,12pt]{amsart}
\synctex=1
%
%%%% packages
\usepackage[utf8]{inputenc}
\usepackage[dvipdfmx]{graphicx,color}%for images
\usepackage{bm}%fonts
\usepackage{tikz-cd}%
\usetikzlibrary{cd}
\usetikzlibrary{calc}
\usepackage{amsmath,amsthm,amstext,amsfonts,amsbsy}% ほぼ必須
\usepackage{amssymb}
\usepackage{latexsym}% ほぼ必須
\usepackage{makecell}%表のセル内で改行するためのパッケージ
\usepackage{algpseudocode,algorithm}%疑似コード用
\usepackage{todonotes}%comments
\usepackage[margin=0.8in]{geometry}
\usepackage{layout}
\usepackage[T1]{fontenc}%font encoding
\usepackage{physics}
\usepackage{braket}%after physics
\usepackage{mathtools,thmtools}
\usepackage{imakeidx}%before hyperref for pagebackref
\usepackage[pagebackref,dvipdfmx]{hyperref}
\usepackage[capitalize]{cleveref}
\hypersetup{
     colorlinks = true,
     citecolor  = blue,
     linkcolor  = blue, 
     urlcolor   = blue, 
}
%\usepackage{pxjahyper}%for hyperref in Japanese
\usepackage{bookmark}
\usepackage{dynkin-diagrams}

%%%% imakeidx
\makeindex
\makeindex[name=not, title=Index, columns=2]
\makeindex[name=sym, title=Symbol, columns=3]
\makeindex[name=ref, title=Ref, columns=3]

\newcommand{\ind}[2]{\emph{#1}\index{1{#2}@{#1}}}
\newcommand{\indset}[3]{$#1 \deq #2 $ \index{0{#3}@$#1$} }
\newcommand{\indse}[2]{{$#1$}\index{0{#2}@{$#1$}}}

%%%%
\usepackage{pgf,tikz,pgfplots}
\pgfplotsset{compat=1.15}
\usetikzlibrary{arrows}



%%%%


%%%% theoremstyle

\theoremstyle{definition}
\newtheorem{Thm}{Theorem}[section]
\newtheorem*{Thm*}{Theorem}
\newtheorem{Def}[Thm]{Definition}
\newtheorem{Def*}{Definition}
\newtheorem{Eg}[Thm]{Example}
\newtheorem*{Eg*}{Example}
\newtheorem{Prop}[Thm]{Proposition}
\newtheorem*{Prop*}{Proposition}
\newtheorem{Note}[Thm]{Note}
\newtheorem*{Note*}{Note}
\newtheorem{Ntc}[Thm]{Notice}
\newtheorem*{Ntc*}{Notice}
\newtheorem{Lem}[Thm]{Lemma}
\newtheorem*{Lem*}{Lemma}
\newtheorem{DefProp}[Thm]{Definition and Proposition}
\newtheorem*{DefProp*}{Definition and Proposition}
\newtheorem{Fact}[Thm]{Fact}
\newtheorem*{Fact*}{Fact}
\newtheorem{Ques}[Thm]{Question}
\newtheorem*{Ques*}{Question}
\newtheorem{Cite}[Thm]{Citation}
\newtheorem*{Cite*}{Citation}
\newtheorem{Conj}[Thm]{Conjecture}
\newtheorem*{Conj*}{Conjecture}
\newtheorem{Rule}[Thm]{Rule}
\newtheorem*{Rule*}{Rule}
\newtheorem{Not}[Thm]{Notation}
\newtheorem*{Not*}{Notation}
\newtheorem{Cor}[Thm]{Corollary}
\newtheorem*{Cor*}{Corollary}
\newtheorem{Rmk}[Thm]{Remark}
\newtheorem*{Rmk*}{Remark}
\newtheorem{Cond}[Thm]{Condition}
\newtheorem*{Cond*}{Condition}
\newtheorem{Conv}[Thm]{Convention}
\newtheorem*{Conv*}{Convention}
%%%% newcommand

%%%logic symbol
\newcommand{\deq}{\coloneqq}

\newcommand{\dbraket}[1]{\hspace{-1.5pt}\braket{\hspace{-2.2pt}\braket{#1}\hspace{-2.2pt}}}

\newcommand{\textcmd}[1]{\texttt{\symbol{"5C}#1}}

%%special sets
\newcommand{\emp}{\varnothing}%emptyset
\newcommand{\C}{\mathbb{C}}%complex number
\newcommand{\Ha}{\mathbb{H}}%quaternion
\newcommand{\F}{\mathbb{F}}%field
\newcommand{\R}{\mathbb{R}}%real number
\newcommand{\Q}{\mathbb{Q}}%rational number
\newcommand{\Z}{\mathbb{Z}}%integer
\newcommand{\N}{\mathbb{N}_{0}}%natural number
\newcommand{\Pj}{\mathbb{P}}%bold p
\newcommand{\vep}{\varepsilon}%varepsilon

%%%%

\newcommand{\mb}[1]{\mathbb{#1}}%blackboard bold (for math mode)
\newcommand{\mcal}[1]{\mathcal{#1}}%
\newcommand{\mf}[1]{\mathfrak{#1}}%

\newcommand{\opn}[1]{\operatorname{#1}}
\newcommand{\catn}[1]{\mathbf{#1}}

\newcommand{\abk}[1]{\langle {#1} \rangle}%angle bracket 
\newcommand{\Abk}[1]{\left \langle {#1} \right \rangle}%angle bracket (auto sizing)
\newcommand{\dabk}[1]{\langle\! \langle {#1}\rangle \! \rangle}%double angle bracket
\newcommand{\Dabk}[1]{\left \langle \! \left \langle {#1} \right \rangle \! \right \rangle}%double angle bracket
\newcommand{\Sbk}[1]{\left[ {#1} ]\right }% square bracket [] (auto sizing)
\newcommand{\Cbk}[1]{\left \{ {#1}\right \}}% curly bracket {} (auto sizing)
\newcommand{\dcbk}[1]{\{\!\!\{ {#1}\}\!\!\}} % double curly bracket {{}} 
\newcommand{\Dcbk}[1]{\left \{\!\! \left \{ {#1} \right\} \!\!\right \}} % double curly bracket {{}} (auto sizing)
\newcommand{\Paren}[1]{\left ( {#1} \right )}%parenthesis () (auto sizing)
\newcommand{\dparen}[1]{(\!({#1})\!)}%double parenthesis
\newcommand{\xto}[1]{\xrightarrow{#1}}
\newcommand{\xgets}[1]{\xleftarrow{#1}}
\newcommand{\hookto}{\hookrightarrow}


%%%% 

%%%% mathabx.sty (font) 
\DeclareFontFamily{U}{matha}{\hyphenchar\font45}
\DeclareFontShape{U}{matha}{m}{n}{
      <5> <6> <7> <8> <9> <10> gen * matha
      <10.95> matha10 <12> <14.4> <17.28> <20.74> <24.88> matha12
      }{}
\DeclareSymbolFont{matha}{U}{matha}{m}{n}

\DeclareFontFamily{U}{mathb}{\hyphenchar\font45}
\DeclareFontShape{U}{mathb}{m}{n}{
      <5> <6> <7> <8> <9> <10> gen * mathb
      <10.95> mathb10 <12> <14.4> <17.28> <20.74> <24.88> mathb12
      }{}
\DeclareSymbolFont{mathb}{U}{mathb}{m}{n}

\DeclareFontFamily{U}{mathx}{\hyphenchar\font45}
\DeclareFontShape{U}{mathx}{m}{n}{
      <5> <6> <7> <8> <9> <10>
      <10.95> <12> <14.4> <17.28> <20.74> <24.88>
      mathx10
      }{}
\DeclareSymbolFont{mathx}{U}{mathx}{m}{n}

%DeclareMathSymbol (from mathabx.sty)
\DeclareMathSymbol{\bigboxslash}{\mathop}{mathx}{"FE}
\DeclareMathSymbol{\bigboxtimes}{\mathop}{mathx}{"D2}
%%%%

%%%% MnSymbol.sty (font)
\DeclareFontFamily{U}{MnSymbolC}{}
\DeclareFontShape{U}{MnSymbolC}{m}{n}{
  <-6> MnSymbolC5
  <6-7> MnSymbolC6
  <7-8> MnSymbolC7
  <8-9> MnSymbolC8
  <9-10> MnSymbolC9
  <10-12> MnSymbolC10
  <12-> MnSymbolC12}{}
\DeclareFontShape{U}{MnSymbolC}{b}{n}{
  <-6> MnSymbolA-Bold5
  <6-7> MnSymbolC-Bold6
  <7-8> MnSymbolC-Bold7
  <8-9> MnSymbolC-Bold8
  <9-10> MnSymbolC-Bold9
  <10-12> MnSymbolC-Bold10
  <12-> MnSymbolC-Bold12}{}

\DeclareSymbolFont{MnSyC}{U}{MnSymbolC}{m}{n}

%%%% DeclareMathSymbol (from MnSymbol.sty)

\DeclareMathSymbol{\tplus}{\mathbin}{MnSyC}{43}
\DeclareMathSymbol{\aplus}{\mathbin}{MnSyC}{190}

%%%% renewcommand




%%%% footnote

\newcommand{\cfootnote}[1]{\footnote{#1}}

\newcommand{\TB}{\mcal{T}_{B_0}}
\newcommand{\TBZ}{\mcal{T}_{\Z,B_0}}
\newcommand{\AffS}{{\mathop{\mcal{A}\!f\!\!f\!}\nolimits}}
\newcommand{\FBZ}{\mcal{F}_{\Z,B}}
\newcommand{\FB}{\mcal{F}_{B}}
\newcommand{\simto}{ 
\mathrel{\raisebox{0.13em}{${\sim}$}}
\kern -0.75em \mathrel{\raisebox{-0.11em}{${\scriptstyle \to}$}}  
}
%%%% from 

%%%% from  https://tex.stackexchange.com/questions/183702/formatting-back-references-in-bibliography-bibtex
\renewcommand*{\backrefalt}[4]{
    \ifcase #1 [Not cited.]%
        \or        [Cited on p.#2.]%
        \else      [Cited on p.#2.]%
    \fi}


\usepackage{mathrsfs}
\usepackage{upgreek}
\numberwithin{equation}{section}
\title{On counting lattice points in some tropical spaces and beyond
}
\author[Y. Tsutsui]{Yuki Tsutsui}
\address{Graduate School of Mathematical Sciences, The University of Tokyo, 3-8-1 Komaba, Meguro-Ku, Tokyo, 153-8914, Japan}
\email{tyuki@ms.u-tokyo.ac.jp}
\date{\today}

\begin{document}

\begin{abstract}
We consider a tropical version of counting lattices point for 
line bundles on tropical curves and integral affine manifold
with a Hessian form.
\end{abstract}
\maketitle
\section{Introduction}

One of main topics of tropical geometry is that 
how to calculate algebraic data on algebraic varieties
from tropical varieties with additional data.
In particular, we may expect that we can calculate
 some invariants of algebraic varieties
which is (generically) invariant for tropicalization
(or degeneration) from tropical variety.

For instance, the tropicalization of a $n$-dimensional
algebraic subvariety of an algebraic split torus 
is a $n$-dimensional polyhedral complex 
(e.g 
\cite[Theorem 3.2.3]{maclaganIntroductionTropicalGeometry2015a}).

We will explain 

\subsection{Tropical Riemann--Roch theorem}

the Hirzebruch--Riemann--Roch theorem and the Grothendieck--
Riemann--Roch theorem are monumental work in
 algebraic geometry (or complex geometry).

Tropical Riemann--Roch theorem for tropical curves 
is proved in \cite{gathmannRiemannRochTheoremTropical2008a}
as an extension of Riemann--Roch theorem for graphs 
\cite{MR2355607}.

The rank $r(D)$ of the linear system of a divisor $D$ 
on tropical curve in 
\cite{gathmannRiemannRochTheoremTropical2008a}
is \emph{not} an invariant of $\mb{T}$-modules\footnote{
See }
but this is truly one of good analog of classical one,
see \cite[Lemma 2.4]{MR2448666}.
The rank
Cartwright defines an invariant 
$h^{0}(\Delta,D)$ for a divisor on tropical complex
which a certain analog of the dimension of
the $0$-th cohomology of a line bundle on algebraic 
variety.
in \cite[Definition 3.1]{MR4251610}.
If $\dim \Delta=1$, then $h^{0}(\Delta,D)=r(D)+1$
\cite[Proposition 3.3]{MR4251610}, and thus $h^{0}(\Delta,D)$
is a generalization of $r(D)+1$ for tropical complexes. 

On the other hand, a tropical analog of Riemann--Roch theorem
for higher dimensional tropical manifolds 
(or tropical complexes) is not formulated
since there hasn't been any good definition of 
Euler characteristic of line bundles on tropical 
manifolds as an extension of the above notions yet.

In \cite{MR3903579},
One of main difference between them is that 
the rank of linear system of tropical curves is
not the invariant of $\mb{T}$-modules.\footnote{}

The other dimensions of $\mb{T}$-modules 
(which are invariant for $\mb{T}$-modules) are defined for 
some authors (e.g. \cite[Definition 2.3]{mikhalkinTropicalCurvesTheir2008a}
and \cite[p.8]{yoshitomi2011generators}) but 
these dimensions have no tropical analog of 
Riemann--Roch formula like 
\cite{,gathmannRiemannRochTheoremTropical2008a}.

One of standard (but very hard) approach is that 
consrtucting 
The homotopical algebra of semimodules is 
studied recently in 
\href{https://arxiv.org/abs/2202.01573}{[2202.01573] Proto-exact categories of modules over semirings and hyperrings}
but the structure of the associated 
stable $\infty$-category of the proto-exact structure
is not studied yet.
\footnote{The nerve of a proto-exact category
forms an exact $\infty$-category 
\cite[Example 7.2.3]{MR3970975} and there exists the
stable hull of it 
\cite{https://doi.org/10.48550/arxiv.2010.04957}.}

From successes of tropical homology \cite{itenbergTropicalHomology2019b}, we 
may expect the existence of transformation from 
line bundles on tropical manifold to complexes of
constructible sheaves whose Euler number 
represents that of line bundle, but we need to 
be careful of that we also can observe that
the derived category $\opn{D}_{c}^{b}(B)$
of constructible sheaves on 
tropical manifold $B$ does not seem like a tropical 
analog of the derived category of coherent 
sheaves on algebraic variety directly.\footnote{One of reasons
comes from the result in \cite{MR2449059,MR2565051} and 
homological mirror symmetry for 
the derived category of coherent sheaves on complex split $n$-torus 
$(\C^{\times})^{n}$ and the derived wrapped Fukaya category of $T^{*}(S^{1})^{n}$
by \cite{MR2822213}.
Therefore, $\opn{D}_c^{b}((S^{1})^{n})$ should behave like 
the derived category of coherent sheaves on $(\C^{\times})^{n}$
but not like of some complex tori.
We will see another reason in \cref{a}.}


\subsection{Lattice polytopes}

We consider about 

Counting points on a certain region is a classical problem.

Since Euler number of line bundle is invariant for 
degeneration (e.g. \cite[p.50]{}) and thus we can calculate
the Euler number (or the dimension of global section)
 of many interesting ample line bundles
by a degeneration to a toric variety or a union of toric
varieties. 

(For instance, let $X\to \opn{Spec}(\Z_{p})$ be a
projective scheme over $\Z_{p}$ and $\mcal{L}$
be a line bundle on $X$ that is flat over $\Z_{p}$.
Then, $\chi_{\kappa(s)}(X_s;\mcal{L}_s)=\chi_{\kappa(\eta)}(X_{\eta};\mcal{L}_{\eta})$

Roughly speaking, this means that
we can calculate the Euler number of line bundles on
a projective scheme over $\Q_p$ from that on a 
projective scheme over $\mb{F}_p$.  

This interpretation works for integral affine manifold 
with singularities $B$ as a lattice point of $B$. 

We want to focus on the following Proposition which is related with
BS=RR problems.

Lattice polytopes naturally appear from tropical toric
varieties as follows: 

\begin{Prop}
Let $P$ be a convex lattice polytope and
$f=\log (\sum_{n\in P\cap \Z^{n}} \opn{exp}(\abk{n,x}))\colon \R^{n}\to \R$ be 
a Laurent polynomial function over the log semiring. Then,
 the differential
$df:\R^{n}\to (\R^{n})^{\vee}$ is an embedding onto $\opn{relint}(P)$
\cite[p.124 Exercise]{MR1301331}.
Besides, the quotient map $q: \R^{n}\to \R^{n}/\Z^{n}$ induces the following
equation.
\begin{align}
\sharp (q(df)\cap 0_{\R^{n}})=\sharp (\opn{int}(P)\cap \Z^{n}). 
\end{align}
Here, $0_{{\R}^{n}}$ is the zero section of $\check{X}(\R^{n})$.
\end{Prop}

\begin{Eg}
Let $f:\R \to \R; x\mapsto \log (1+e^{x})$ be a soft 
plus function. When we take a dequantization of $f$,
we get a ReLU $f^{\opn{trop}}(x)=\max\{0,x\}$.
The differential is a sigmoid function 
$df(x)=\frac{e^{x}}{1+e^{x}}$ 
\footnote{softmax functions also are appreared 
naturally in a similar way when we consider tropical projective spaces.}.
\end{Eg}

We can naturally extend this proposition for tropical projective toric 
varieties and then the extension map is called the \emph{tropical moment map}
$\mu_{P}^{\opn{trop}}: X_{P}^{\opn{trop}}\simto P$ 
\cite[Definition 2.1 (2)]{MR2428356}.

Instead of moment map for Hamiltonian action of Lie group, we
need not assume convex lattice polytope is not Delzant.

This is a tropical analog of algebraic moment map 
(e.g. \cite[\textsection 12.2]{coxToricVarieties2011a}) for toric varieties.

We can generalize the above proposition as follows;
\begin{align}
\sharp (q(\mu_P^{\opn{trop}}) \cap 0_{X_P^{\opn{trop}}})
=\sharp (P \cap \Z^{n}).
\end{align}

From construction of tropical moment map for lattice polytope,
it seems that generic rational section of line bundle on 
toric variety (over log semifield) generate the basis of
.


These data is very similar with elements of 
$\opn{CaDiv}^{\infty}(X_{P}^{\opn{trop}})\deq H^{0}(X_{P}^{\opn{trop}};
\mcal{A}^{0,0}_{X_{P}^{\opn{trop}}}/\mathcal{O}^{\times}_{X_{P}^{\opn{trop}}})$.

\footnote{Be careful that this data is not an element
of $\opn{CaDiv}^{\infty}(X_{P}^{\opn{trop}})$ since
 the former collection of smooth function is not constant
at the neighborhood of the corner of $X_{P}^{\opn{trop}}$,
i.e., the fixed point of big the tropical torus orbits.
We need extend $\mcal{A}^{0,0}_{X_{P}^{\opn{trop}}}$
suitably to treat as a truly analog of Cartier data.}



$\opn{CaDiv}^{\infty}(X_{P}^{\opn{trop}})$ is a
smooth version of the group of tropical Cartier divisor
$\opn{CaDiv}(X_{P}^{\opn{trop}})$ 
\cite[Definition 4.2]{jellLefschetzTheoremTropical2018a}.

This observation gives an ex

Another important point is that $q\circ df$ is also a Lagrangian
submanifold of the torus fibration $\check{X}(\R^{n})\deq \R^{n}\times (\R^{n})^{\vee}/(\Z^{n})^{\vee}$.


These Lagrangian submanifolds are called
\emph{Lagrangian sections} \index{Lagrangian section@Lagrangian section}
which is well-studied in the field of homological 
mirror symmetry 
\cite{kontsevichAffineStructuresNonArchimedean2006a,MR2452307}.
In particular, Lagrangian sections are mirror of line bundles on
mirror varieties. 
In this paper, we consider a 
This gives an interesting insight which is inspired from
Strominger--Yau--Zaslow conjecture 
(SYZ conjecture for short).
(See \cref{rmk: floer-coherent-problem} 
for detail about it.)

The following question is not formulated as a mathematical 
conjecture since we don't know the definition of Todd 
class of $n(\geq 3)$ dimensional tropical manifolds,
but we write it for showing a slogan of formulation 
of a tropical analog of Riemann-Roch theorem.

\begin{Conj}[{Tropical Mirror Riemann--Roch Problem}]
Let $B$ be a compact tropical manifold and $\mcal{L}$
be a line bundle on it. Suppose $B$ has the Todd class
$\opn{td}(B) \in \mb{H}^{\bullet}(B;\Omega_B^{\bullet})$.
 Let $s$ be a "generic"
extended smooth Cartier data which represents of $\mcal{L}$
and its Floer graded $\Z$-module $\opn{MF}^{\bullet}(s)$
of it. Then,

\begin{align}
\chi(B,s)  \deq \chi(\opn{MF}^{\bullet}(s))=\int_B \opn{ch}(\mcal{L})\opn{td}(B).
\end{align}

In particular, when $\mcal{L}=\mcal{O}_B$ then
$\chi(\opn{MF}^{\bullet}(s))=\chi_{\opn{top}}(B)$?
\end{Conj}

Here, $\opn{ch}(\mcal{L})$ is a tropical analog of 
the total Chern class of $\mcal{L}$.
On the other hand, $\opn{td}(B)$ is
"a tropical analog of Todd class"
but we don't know the definition of it. 
The definition of higher tropical Chern class of $B$
is mentioned in \cite[5.3]{mikhalkinTropicalGeometryIts2006},
but we don't understand the weight of the higher Chern class 
of $B$ is not written explicitly.
(See \cite[Definition 3.20]{shawTropicalSurfaces2015a} for 
the weight of top Chern class of $B$.)


In this paper, we give an answer for compact tropical curves 
and integral affine manifolds with Hessian form
as test plays of the above question:

\begin{Thm} \label{thm: main}
Let $B$ be a compact tropical curve $C$ (resp. integral 
affine manifold $B$
with a Hessian form) and $s=\{(U_i,f_i)\}_{i\in I}$ a polite $C^{1}$-
Cartier data on $C$ (resp. $B$). Then,
\begin{align}
\chi(C,s)=c_1(D_s)+c_1(-K_C), \quad \chi(B,s)=\frac{c_1(D_s)^{n}}{n!}
\end{align}
where $\chi(C,s)$ (resp. $\chi(B,s)$) is the number of weighted lattice
points on $s$ (see \cref{def: weighted_lattice_points}) and $D_s$ is the divisor class of $s$.
\end{Thm}

The condition of politeness is that every cotangent vector 
is not in hyperplanes defined by the span of microsupport of
cotangent sheaf of $C$. 


We note that this theorem has already essentially proved for some
literature as a trivial corollary of homological mirror symmetry 
in \cite{MR4301560} and \cite{auroux2022lagrangian} (see \cref{rmk: curve_mirror} and \cref{rmk: integral_mirror} 
for more details) but we reprove it in tropical  
geometrical setting as a test play of another approach 
for a tropical analog of Riemann-Roch formula for 
tropical spaces. This approach is different
from classical approach such like 
\cite{gathmannRiemannRochTheoremTropical2008a,MR4251610} but
a toy model of homological mirror conjecture in the sense
of \cite[{\textsection 7}]{auroux2022lagrangian}.

\begin{Rmk}
We expect our approach works for various style 
of tropical spaces and develop a certain tropical 
analog of Ehrhart theory. 
\end{Rmk}

\textit{Acknowledgments:}

This work was supported by JSPS KAKENHI 
Grant Number 21J14529.

\begin{Note}[Guide for reading this paper]
  In order to reveal the relationships between our study
  and other topics, we cannot avoid introducing various
  notions. However, our results are almost elementary
in the case of tropical curves,
  so we write down a shortcut course of this paper as below:
\end{Note}

\begin{Note}
  We also impose the following assumption for simplicity in this paper:

  Any topological space is locally compact and locally
  contractible except schemes.


  We mainly use min-plus algebra.

  \footnote{We write down the reason of this in \cref{rmk: mix vs max}.}


\end{Note}

\section{The cohomological local Morse datum}
In this section we recall some elementary result
for cohomological local Morse data from
\cite{MR2031639,MR1299726,MR4294126}.

\subsection{Sheaf theory on locally compact 
Hausdorff spaces.}
The notion and symbols for 
sheaf theory follows from \cite{MR1299726}.
Let $X$ be a topological space and $Z$ a closed subset of $X$.





The following stalk complex (or graded module) is the most important notion
for this paper which is naturally appeared from sheaf theory
which is called "local Morse datum" in \cite[p.271]{MR2031639} 
or microlocal stalk in \cite{MR4132582}.
\begin{Def}[{Cohomological local Morse datum}]
Let $f:X\to \R$ be a $C^{1}$-function on a smooth manifold $X$ and has 
at most one critical point $x$.
\begin{align}
\opn{LMD}(\mcal{F},f,x)\deq R\Gamma_{\{f\geq f(x)\}}
(\mcal{F})_x, \quad \opn{MF}^{\bullet}(\mcal{F},f,x)\deq 
H^{\bullet}(\opn{LMD}(\mcal{F},f,x)).
\end{align}
\end{Def}

\begin{Rmk}
We don't assume $f$ satisfies some 
"Morse condition" like \cite{MR2031639,MR4132582} since we need to use some $C^{1}$-function
which is \emph{not} a Morse function in the sense of 
stratified Morse theory. 
We also note that we need to use 
the graded module $\opn{MF}^{\bullet}(\mcal{F},f,x)$ instead of 
the complex $\opn{LMD}(\mcal{F},f,x)$ since we need to
forget the differential.
\end{Rmk}

\begin{Eg}[Morse function]
  We
\end{Eg}

\subsection{Polite condition}

Let $M$ be a free $\Z$-module of finite rank and 
$M_{\R}\deq M\otimes_{\Z}\R$.

\begin{Def}[{\cite{MR4294126}}]

\end{Def}

Every PL set is subanalytic, and thus 
we can use the theory of microsupport of sheaves
effectively.
Let $\mcal{F}$ be a 

We will develop this without condition 

\begin{Eg}



Let $S$ be a rational PL set in $\R^{n}$ and
$f$.

The remarkable feature gives a certain analog
of Kodaira vanishing theorem. See \cref{eg: kodaira}.
\end{Eg}



\begin{Rmk}[{A vanishing properties
for some good fan}]
$f\colon M_{\R}\to \R$ be a strictly convex
$C^{1}$-function on $M_{\R}$ with the minimum point $x$
 and $S$ be the support of 
a polyhedral complex $\Sigma$ whose contains $x$.

One of trivial but remarkable point is that
\begin{align}
\opn{MF}^{\bullet}(A_S,f,x)\not\simeq (-1)^{n}
\opn{MF}^{\bullet}(A_S,-f,x)
\end{align}

\end{Rmk}

\begin{Def}[{politeness}]

\end{Def}

\begin{align}
\opn{MF}^{\bullet}(S,f;A)\deq 
\bigoplus_{(x,m)\in \opn{Crit}(f,S)} \opn{MF}^{\bullet}(A_S,f+\check{m},x)
\end{align}


\section{Proof for compact tropical curves}
In this section, we prove \cref{thm: main} for 
compact tropical curves
(see \cite{mikhalkinTropicalCurvesTheir2008a} 
for the definition of tropical curves).
Our proof is highly effected from 
\cite{knill2012graph,MR2676658,auroux2022lagrangian}.
\footnote{Our proof idea comes from 
\cite{knill2012graph,MR2676658} at first.  
The current proof is modified after reading 
\cite{auroux2022lagrangian}.}.
We also follow the notation used in 
\cite{auroux2022lagrangian}.

\begin{Def}[{Continuous section}]
Let $C$ be a tropical curve and 
$C_{0}\deq \{x\in C\mid \opn{val}(x)=2\}$.

\begin{align}
\check{X}(C_0)\deq T^{*}C_0/T^{*}_{\Z}C_0, \quad 
X_0(C)\deq \check{X}(C_0)\sqcup_{i,s}C
\end{align}

\end{Def}

\begin{Thm}


\end{Thm}

\begin{Rmk} \label{rmk: curve_mirror}
When $C$ is a trivalent metric graph, 
our Lagrangian sections are 
almost the same with that of fiberwise wrapped Fukaya category 
$\mcal{F}(M)$ \cite[3.1]{auroux2022lagrangian}.
Here, $M$ is the
union of $\C P^{1}$'s whose intersection complex is $C$, 
i.e., every edge of $C$ corresponds
$\C P^{1}$ and every trivalent vertex of $C$ corresponds
to the intersection points of some three $\C P^{1}$'s.
The mirror manifold of $M$ is an algebraic curve $X_K$
from Mumford's construction using the dual intersection
complex $C$.

The local model of $M$ comes from the critical locus
of the potential function of a (A-side) Landau--Ginzburg 
model $(\C^{3},-xyz)$. There exists a (Orlov) functor
from $\opn{Fuk}(\C^{3},-xyz)$ 
to $\opn{wFuk}((\C^{\times})^2)$ which is compatible
with the pushforward 
$i_*:\opn{Perf}(V(1+x_1+x_2))\to \opn{Perf}((K^{\times})^{2})$. 
of the inclusion $i:V(1+x_1+x_2)\hookto (K^{\times})^2$
via mirror functors (see \cite[2.]{auroux2022lagrangian}
for more details). 

The mirror space $M$ of $X_K$ can be considered as a 
topological compactification of $\check{X}_0(C)$.

By a slight modification of $\check{X}_0(C)$, every 
stratified smooth Lagrangian section of $\check{X}(C)$ 
can be considered as an object of $\mcal{F}(M)$

The main difference between our setting and 
that of \cite{auroux2022lagrangian} 
is the definition of graded space $\opn{CF}(C,s)$. 
Our approach has no good definition 
of $m^{d}$-operations but we define $\opn{CF}(C,s)$ 
without Hamiltonian
perturbations.
\end{Rmk}

The stratified torus fibration of Riemann surface is 
closely related with pants decomposition, and thus
Auroux--Efimov--Katzarkov in \cite{auroux2022lagrangian} expect the mirror symmetry 
for projective hypersurfaces since they have higher 
dimensional pants decomposition \cite{MR2079993} along
tropical hypersurfaces.
As a toy model of this question, we also pose a 
following question as a higher version of the 
above correspondence.
\begin{Ques}
Is there exists an analog of the correspondence of the 
localization of index for
hypersurfaces of projective space and pants decomposition 
in \cite{MR2079993}?
\end{Ques}
We also expect that the above question has a generalization
for more general algebraic varieties.
\begin{Def} \label{def: weighted_lattice_points}

\end{Def}

By imitating of the objects of fiberwise wrapped Fukaya category 
in \cite{auroux2022lagrangian}, we can define a tropical 
analogue of vector bundles.
\begin{Eg}[{Multi-section}]

This is a tropical analog of 
\end{Eg}

\subsection{Relationships between a localization of 
index on Riemann surfaces.}

\section{Proof for integral affine manifold with Hessian form.}


We write some background from 


\begin{Def}

\end{Def}

\begin{Eg}

\end{Eg}

Every integral affine 
manifold is a tropical manifold and every tropical manifold is 
a rational polyhedral space, so we use the theory of 
tropical homology.
From now on, we follow about tropical homology 
from \cite{MR3903579,gross2019sheaftheoretic}.



\begin{Def}[Smooth Cartier data]
  A \ind{very polite smooth Cartier data}{very polite smooth Cartier data} $s=\{f_i\}$ is an element of
  $H^{0}(B,\mcal{A}_B^{0,0}/\mcal{O}_{B}^{\times})
    \simeq H^{0}(B,\mcal{Z}^{1}_B/\mcal{F}_{\Z,B}^{1})$ such
that every $f_i$ is polite.
\end{Def}


There exists a natural isomorphism $\Omega_X^{n-p}[n]\to 
\mcal{D}(\Omega_{X}^{p})$ where 
$\mcal{D}(\mcal{F}^{\bullet})
=R\mcal{H}om(\mcal{F};\upomega_X^{\bullet})$
\cite[Theorem 6.2]{gross2019sheaftheoretic}.

In particular, the natural isomorphism $\Omega_X^{n}[n]\to 
\mcal{D}(\Omega_{X}^{0})=\upomega_B^{\bullet}$
can be considered as the element of $H_{n,n}^{\opn{BM}}(X;\Z)$ 
from definition. 
We call this element as the \emph{fundamental class} of $X$ 
and it denotes by $[X]$. The fundamental class of tropical
manifold was defined from the locally constant function on 
$X_{\opn{reg}}$. (See \cite[Definition 4.8]{jellLefschetzTheoremTropical2018a}.)
If $X=\opn{pt}$, $H_{0,0}^{\opn{BM}}(\opn{pt};\Z)=\opn{End}(\Z)\simeq \Z$.
Therefore, $H^{n,n}(X;\Z)\xto{\cdot \cap [X]} H_{0,0}^{\opn{BM}}(X;\Z)
\xto{f_!} H_{0,0}^{\opn{BM}}(\opn{pt};\Z)$ defines
the 
When $B$ is an integral affine manifold of dimension $n$, 
$\upomega^{\bullet}_{B}\simeq \opn{or}_{B}^{\Z}[n]$ 
\cite[]{}.
and the fundamental class of $B$ is a $\Z$-orientation of $B$
when $B$ is orientable.

\begin{Rmk}
When $s=df$, $\chi(B,s)=\chi_{\opn{top}}(B)=0$ is 
truly a special case of Poincar\'e--Hopf theorem for $B$.
We also note there exists another tropical analog of Poincar\'e--Hopf theorem
  \cite{rau2020tropical}. This analogue is about tropical Euler characteristic
  $\chi_{\opn{trop}}(B)\deq \chi(\mb{H}^{\bullet}(B;\Omega_B^{\bullet}))$
  but not for topological Euler characteristic $\chi_{\opn{top}}(B)$.
\end{Rmk}

\begin{Def}
Let $B,B'$ be an integral affine manifold.
\end{Def}

\begin{Rmk}
Our condition of \'etale is different from that of 
\cite[Definition 1.1]{grossMirrorSymmetryLogarithmic2006a}.

If $f\colon B\to B'$ be an affine local diffeomorphism,
then $X(f)\colon X(B) \to X(B')$ is a local isomorphism of
complex manifolds.
\end{Rmk}

\begin{Prop}
Let $f:B'\to B$ be a finite \'etale covering of a compact integral affine 
manifold $B$ of dimension $n$. Then,
(i) $\chi(B',f^{*}(s))=\opn{deg}(f)\chi(B,s)$

(ii) (Projection formula) 
$\frac{c_1(f^{*}(D_s))^{n}}{n!}=\opn{deg}(f)\frac{c_1(D_s)^{n}}{n!}$

(iii) If $B$ has a Hessian form, then $\chi(B,s)=\frac{c_1(D_s)^{n}}{n!}$
\end{Prop}

\begin{proof}
(i) We can see this from definition.

(ii) Since $f^{!}=f^{-1}$ 
\cite[]{iversenCohomologySheaves1986a}
, $\opn{deg}(f)$ is the same with degree of 
mapping of closed manifolds.

(iii) If $B$ has a Hessian form, then there exists a tropical tori $T$
and an \'etale cover $f\colon T\to B$ by Cheng-Yau's result
\cite{MR714338}. Since $f^{*}(s)$ is a Lagrangian section
of $\check{X}(T)$, $\chi(T,f^{*}(s))$ is an intersection
number of
\end{proof}

\begin{Rmk} \label{rmk: integral_mirror}
This theorem also can be considered as a special case of \cite{MR4301560}.
\end{Rmk}

\begin{Rmk}

According to \cite[5.3]{mikhalkinTropicalGeometryIts2006},
the support of $k$-th Chern class of tropical manifold is 
the $k$-skeleton of it.
We don't know the definition of higher Chern class of tropical manifold $B$ except
$n=1$ or $n=\dim B$. 
On the other hand, $\opn{ch}(B)$ should be trivial when $B$ is
an integral affine manifold, since $B$ has empty
$k$-skeleton except $k=n$.
Thus, $c_{k}(B_0)$ should be $1$ except $k=0$, 
i.e., $\opn{td}(B_0)=1$. 

We already see $\hat{\mcal{A}}(X(B_0))=\opn{td}(X(B_0))=1$.
    From semi-flat SYZ mirror symmetry, the above conjecture should be true. In the case of tropical tori, see \ref{}.
\end{Rmk}

\begin{Rmk}
If $B_0$ is not a Hessian manifold, then we cannot apply the above proof
but we can also prove for closed integral affine surface by the same logic.
See \cite{MR1422337} for tropical primary Kodaira surfaces.
If Markus conjecture is true, $\widetilde{X(B_0)}\simeq \C^{n}$
for some $n\in \N$.
\end{Rmk}

\section{For more examples}



\begin{Def}
Let 
\end{Def}
We didn't formalize this 
We restrict for the condition of Lagrangian section and 
give a certain extension of the main theorem.
In this proof,
\begin{Cor}[{K\"unneth formula}]
Let $X,Y$ be a compact tropical curve or integral affine manifold.
Let $s$ and $s'$ be (fiberwise) Lagrangian section of $X$ and $Y$ 
satisfying the condition of cohomological version of a Milnor 
fibration \cite[Assumption 1.1.1]{MR2031639} locally.
Then, the induced external tensor product 
$s\boxtimes s'\deq \opn{pr}_X^{*} (s)\times \opn{pr}_Y^{*}(s)$
has the following equations:
\begin{align}
\chi(X\times Y,s\boxtimes s')=\chi(X,s)\chi(Y,s')
\end{align}

\end{Cor}
\begin{proof}
From sheaf theoretic Thom-Sebastiani theorem for constructible sheaves 
\cite[Corollary 1.2.1]{MR2031639}, we get an isomorphism of 
graded spaces;

\begin{align}
\opn{CF}^{\bullet}(X\times Y;s\boxtimes s') 
& =\bigoplus_{(v,w)\in s_0\cap s\boxtimes s'}
(R^{\bullet}_{\{f_v\boxplus g_w\geq 0\}}\Z_{U_{(v,w)}})_{(v,w)} \\
& =\bigoplus_{(v,w)\in s_0\cap s\boxtimes s'}
(R^{\bullet}_{\{f_v\geq 0\}}\Z_{U_v})_v
\otimes_{\Z} (R^{\bullet}_{\{g_w\geq 0\}}\Z_{U_w})_w \\
& \simeq
\opn{CF}^{\bullet}(X,s)\otimes_{\Z} \opn{CF}^{\bullet}(Y,s') 
\end{align}

\end{proof}

\begin{Rmk}
In the proof of the above corollary, we didn't use anything
of special conditions from tropical curves and integral
affine manifolds.
\end{Rmk}

We will develop this paper for integral affine manifold
with singularities. As an example we also gives the 
following analog for (ADE type) tropical Kummer surfaces.

\begin{Prop}[{For the proof for Tropical Kummer surfaces}]
Let $L=q(L')$ be the image of a Lagrangian 
section $L'$ of $\check{f}_{T}:\check{X}(T)\to T$ such 
that $L'$ is invariant under the negation map. 
\end{Prop}
\begin{proof}

\end{proof}


\section{Future works}
We need to 
More difficult problem is about what is the differential $m_1$ or higher multiplication $m_d$ for tropical manifolds.
If such $m_d$ exists, this should be determined by the contribution of the moduli space of tropical Morse tree for tropical manifold.
\begin{Ques}
  Is there a Morse $A_{\infty}$-precategory for tropical manifolds?
\end{Ques}

\begin{Rmk}
Our approach is similar with the theory of constructible sheaves 
on algebraic varieties.
We expect
\end{Rmk}

\appendix

\section{The local index}

\begin{Eg}
  If $S_1=\set{v},S_2=\set{w}$ and $\mcal{F}_1=\Z_V, \mcal{F}_2=\Z_{W}$,
  the Thom-Sebastiani theorem gives
  a certain K\"unneth formula for
  $\opn{CF}^{\bullet}(f_1\dot{+}f_2,\Z_{V\times W})$:
  \begin{align}
    \opn{CF}^{\bullet}(f_1\dot{+}f_2,\Z_{V\times W})
    \simeq \opn{CF}^{\bullet}(f_1,\Z_{V})
    \otimes_{\Z} \opn{CF}^{\bullet}(f_2,\Z_{W}), \quad
    \opn{ind}_{(v,w)}(f_1\dot{+}f_2)=\opn{ind}_v(f_1)\cdot \opn{ind}_v(f_2).
  \end{align}
  From instance, if $f_1(x)=x_1^{2}+\cdots x_{m}^{2}$
  and $f_2(x)=-x_{m+1}^{2}-\cdots - x_{m+n}^{2}$ then we have

  \begin{align}
    \opn{CF}^{\bullet}(f_1\dot{+}f_2,\Z_{\R^{n+m}})
    \simeq \tilde{H}^{\bullet-1}(S^{n-1};\Z)
\simeq H^{\bullet}_c(B^{n};\Z)
    \simeq \Z[-\opn{ind}_{\mathrm{Morse}}(f_1\dot{+}f_2,0)]
  \end{align}
  where $\opn{ind}_{\mathrm{Morse}}(f,0)$ is the Morse index
  of a Morse function $f$ at the origin.
This is compatible with classical Morse homology for Morse-Smale functions
on closed Riemannian manifolds.
\end{Eg}


\appendix

\section{Radiance obstruction and geometric quantization}

In this section, we mention about 

\section{Etc of HMS and SYZ}

\begin{Rmk}[{Reason why Calabi--Yau's condition is needed}]



We note Abouzaid constructed ($\Z/2\Z$-graded)
Fukaya categories of higher genus closed surfaces in
\cite{MR2383898}. 
\end{Rmk}

\subsection{Wrapped Fukaya category for toric variety}
$\opn{Fuk}(;\Lambda_{\Sigma})$
\cite[Theorem 3.4]{MR2871160}

\bibliography{lattice_points_surface}
\bibliographystyle{halpha}

\printindex

\end{document}