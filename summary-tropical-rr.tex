\documentclass[uplatex,dvipdfmx,12pt]{jsarticle}

%%%% packages
\usepackage[utf8]{inputenc}
\usepackage[dvipdfmx]{graphicx,color}%for images
\usepackage{bm}%fonts
\usepackage{tikz-cd}%
\usetikzlibrary{cd}
\usetikzlibrary{calc}
\usepackage{amsmath,amsthm,amstext,amsfonts,amsbsy}% ほぼ必須
\usepackage{amssymb}
\usepackage{latexsym}% ほぼ必須
\usepackage{makecell}%表のセル内で改行するためのパッケージ
\usepackage{algpseudocode,algorithm}%疑似コード用
%\usepackage{todonotes}%comments
\usepackage[margin=0.8in]{geometry}
\usepackage{layout}
\usepackage[T1]{fontenc}%font encoding
\usepackage{physics}
\usepackage{braket}%after physics
\usepackage{mathtools,thmtools}
\usepackage{imakeidx}%before hyperref for pagebackref
\usepackage[dvipdfmx]{hyperref}
\usepackage[capitalize]{cleveref}
\hypersetup{
     colorlinks = true,
     citecolor  = blue,
     linkcolor  = blue, 
     urlcolor   = blue, 
}
\usepackage{mathrsfs}
\usepackage{upgreek}
\numberwithin{equation}{section}

\usepackage{titlesec}

\titleformat*{\section}{\large\bfseries}

%%%% theoremstyle

\theoremstyle{definition}
\newtheorem{theorem}{定理}[section]
\newtheorem*{theorem*}{定理}
\newtheorem{definition}[theorem]{Definition}
\newtheorem{definition*}{Definition}
\newtheorem{example}[theorem]{Example}
\newtheorem*{example*}{Example}
\newtheorem{proposition}[theorem]{Proposition}
\newtheorem*{proposition*}{Proposition}
\newtheorem{Note}[theorem]{Note}
\newtheorem*{Note*}{Note}
\newtheorem{Ntc}[theorem]{Notice}
\newtheorem*{Ntc*}{Notice}
\newtheorem{lemma}[theorem]{Lemma}
\newtheorem*{lemma*}{Lemma}
\newtheorem{Fact}[theorem]{Fact}
\newtheorem*{Fact*}{Fact}
\newtheorem{question}[theorem]{Question}
\newtheorem*{question*}{Question}
\newtheorem{conjecture}[theorem]{予想}
\newtheorem*{conjecture*}{予想}
\newtheorem{Rule}[theorem]{Rule}
\newtheorem*{Rule*}{Rule}
\newtheorem{notation}[theorem]{Notation}
\newtheorem*{Not*}{Notation}
\newtheorem{corollary}[theorem]{Corollary}
\newtheorem*{corollary*}{Corollary}
\newtheorem{remark}[theorem]{Remark}
\newtheorem*{Rmk*}{Remark}
\newtheorem{condition}[theorem]{条件}
\newtheorem*{condition*}{条件}
\newtheorem{Conv}[theorem]{Convention}
\newtheorem*{Conv*}{Convention}
\newtheorem{observation}[theorem]{Observation}
\newtheorem*{observation*}{Observation}
\newtheorem{expectation}[theorem]{メタ予想}

% newcommand
\newcommand{\deq}{\coloneqq}
\newcommand{\opn}[1]{\operatorname{#1}}
\newcommand{\catn}[1]{\mathbf{#1}}
\newcommand{\beforesection}{\vspace{-20pt}}
\newcommand{\aftersection}{\vspace{-10pt}}

\title{論文内容の要旨
}
%\author[Y. Tsutsui]{Yuki Tsutsui}
% \address{Graduate School of Mathematical Sciences,
% The University of Tokyo, 3-8-1 Komaba, Meguro-Ku,
% Tokyo, 153-8914, Japan}
% \email{tyuki@ms.u-tokyo.ac.jp}
\date{}

%layout

%package

\usepackage{lineno}

\begin{document}
\setlength{\baselineskip}{-5pt}
\setlength{\parskip}{2pt}

\linenumbers

\maketitle
{\large
\noindent
論文題目:

On graded modules associated with line bundles on 
tropical curves and integral Hessian manifolds

(トロピカル曲線と整Hesse多様体上の直線束に付随した
次数付き加群について)
}

\vspace{5pt}

\noindent
{\large
氏名: 筒井 勇樹
}

\vspace{10pt}

\setlength{\baselineskip}{-3pt}
\setlength{\parskip}{2pt}

本論文の目的はトロピカル幾何学上の直線束を表現する
良い条件をもつ$C^{\infty}$級Cartierデータに対して次数つき加群を
定義すること
と(2)その定式化から自然と提起される Riemann--Roch 型の定理の
新たな類似をトロピカル曲線と Hesse 計量をもつ整アフィン多様体に対して
証明する。本論文の定式化におけるトロピカル曲線上の
直線束の Riemann--Roch 型の定理の類似は Gathmann--Kerber
によるトロピカル Riemann--Roch の定理とは異なる。

\beforesection

\section{Riemann--Rochの定理のトロピカル幾何的類似の研究の背景}

\aftersection

トロピカル幾何とは、ある条件を満たす多面複体の台空間 $S$ とその上の
局所的に整数係数アフィンな関数のなす層 
$\mathcal{O}_S^{\times}$
の組を代数多様体の類似物とみなして研究する分野である
\cite{mikhalkinTropicalEigenwaveIntermediate2014a,
gross2019sheaftheoretic}。
そのような空間をトロピカル多様体と呼ぶ。
一次元トロピカル多様体をトロピカル曲線と呼び、
有限距離グラフは標準的なトロピカル曲線の構造を持つ。
トロピカル多様体であり境界なし微分多様体の構造を自然にもつ空間を
整アフィン多様体と呼ぶ。
Hesse 計量は整アフィン多様体における
K\"ahler 計量の類似である。

コンパクトトロピカル曲線 $C$ 上の Cartier 因子 
$D\in \opn{CDiv}(C)$
に関する Riemann--Roch の定理は
Gathmann--Kerber
\cite{gathmannRiemannRochTheoremTropical2008a} によって 
Baker--Norine \cite{MR2355607}による
グラフの Riemann--Roch の定理の一般化として
次のような等式として証明された。
\begin{align} \label{equation-tropical-rr}
r(D)-r(K_C-D)=\opn{deg}(D)+\chi_{\mathrm{top}}(C)
\end{align}
ここで、$r(D)$ とは、$D$ の因子の線形系の階数であり、
$K_C$ は$C$ の標準因子、$\chi_{\mathrm{top}}(C)$ は
$C$ の位相的 Euler 数である。
この等式を高次元の場合に一般化することは自然な考えであるが
両辺ともに技術的な困難が存在する。代数幾何における
Hirzebruch--Riemann--Roch の定理では、
左辺は因子に付随する直線束の層コホモロジーの
Euler 数に、右辺は 直線束の Chern 類と Todd 類の
カップ積で表現されるが、トロピカル幾何においては
直線束の層コホモロジーと Todd 類の類似物は一般には
定義されていない。特に、
トロピカル直線束は Abel 群のなす層ではない。
一方で、数多くの研究結果から
トロピカル多様体の構造層の Euler 数の類似は
トロピカル多様体上の定数層の Euler 数であると
考えられている。

\beforesection

\section{SYZ予想および本研究の主結果}

\aftersection

本研究では、Strominger--Yau--Zaslow 予想 (SYZ予想)
のアイデアと超局所層理論\cite{MR1299726}を
利用することで、
境界を持たないトロピカル多様体上の直線束を表現する 
permissible な滑らかな因子に対して
Floer 複体もしくは Morse 複体の類似の
次数付き加群を定義し、その次数つき加群の 
Euler 数を 直線束の Euler 数のトロピカル類似とする。

SYZ 予想\cite{stromingerMirrorSymmetryTduality1996}
とは、Calabi--Yau 多様体 $\check{X}$ の
ミラー Calabi--Yau 多様体 $\mathcal{X}$ が、
ある位相多様体 $B$ への(特異ファイバーを許す)
特殊 Lagrange トーラスのファイブレーション
$\check{f}_{B}\colon \check{X}\to B$ の
双対トーラスファイブレーション $f_{B}\colon 
\mathcal{X}\to B$
として得られるという弦理論に関わる数理物理学における
予想であり、この予想の数学的定式化は、
$\check{X}$の導来深谷圏 $\mathrm{fuk}(\check{X})$ から
 $\mathcal{X}$ 上の連接層の導来圏$\mathrm{coh}(\mathcal{X})$
への擬同型関手$\Phi_{\mathrm{SYZ}}$ を誘導すると
期待されている。この$\Phi_{\mathrm{SYZ}}$を
SYZ変換と呼ぶ。
$\Phi_{\opn{SYZ}}$が
存在するならば、Floer コホモロジーと Ext 群との間に
\begin{align}
\opn{HF}^{\bullet}(\mathscr{L}_1,\mathscr{L}_2)
\simeq \opn{Ext}^{\bullet}_{\mathcal{O}_{\mathcal{X}}}(\Phi_{\opn{SYZ}}(\mathscr{L}_1),
\Phi_{\opn{SYZ}}(\mathscr{L}_2))
\end{align}
があることが導かれる。ここで
$\mathscr{L}_1$, $\mathscr{L}_2$ は、$\catn{fuk}(\check{X})$
の対象であり、コンパクト Lagrange 部分多様体といくつかの追加
データによって構成されている。
% $[L_1]$, $[L_2]$ をそれぞれ
% $\mathscr{L}_1$, $\mathscr{L}_2$ の Lagrange 部分多様体
% のホモロジー類だとすると、定義より
% \begin{align}
% \chi(\opn{HF}^{\bullet}(\mathscr{L}_1,\mathscr{L}_2))=
% (-1)^{\frac{n(n-1)}{2}} \sharp ([L_1]\cap [L_2])
% \end{align}
% であることが導かれる。
以上は Calabi--Yau 多様体に関する
ホモロジー的ミラー対称性予想についてだが、
Mumford 曲線に一般化したもの\cite{auroux2022lagrangian}
も考察されている。

$B$ を
($\pi_2(B)=0$となる)
コンパクト整アフィン多様体とする。このとき、$B$ の
構造から誘導される特殊Lagrangeトーラス束 
$\check{X}(B)\to B$ と 
双対トーラス束 $\mathcal{X}(B)\to B$ が定義される。
$B$ 上のトロピカル直線束は、$H^{1}(B;\mathcal{O}^{\times}_B)$
によってパラメトライズされており、滑らかな関数のなす層 
$\mathcal{C}^{\infty}(B)$ への単射 
$\mathcal{O}_{B}^{\times}\to \mathcal{C}^{\infty}(B)$ が誘導
する $H^{0}(B;\mathcal{C}^{\infty}(B)/\mathcal{O}_{B}^{\times})$
から $H^{1}(B;\mathcal{O}^{\times}_B)$ への全射を持つ。
$H^{0}(B;\mathcal{C}^{\infty}(B)/\mathcal{O}_{B}^{\times})$
の元は、整アフィン多様体上のトロピカル Cartier 因子と
類似するものであり、本論文では区別のため$C^{\infty}$級
Cartier データと
読んでいる。このとき、次の同型
\begin{align} \label{equation-lagrangian-section}
H^{0}(B;\mathcal{C}^{\infty}(B)/\mathcal{O}_{B}^{\times})
\simeq \{s\colon B\to \check{X}(B)\mid 
\check{f}_{B}\circ s=\opn{id}_B, \,
\opn{Im}(s) \text{ が Lagrange 部分多様体} \}
\end{align}
が成り立つ。(\ref{equation-lagrangian-section}) の
右辺に属する元を Lagrange 切断と呼び、
自然と $\catn{fuk}(\check{X}(B))$ の対象
$\mathscr{L}_s$ とすることができ、
SYZ 変換によって$\mathcal{X}(B)$ 上の直線束へと
変換される。同型 (\ref{equation-lagrangian-section})
より Lagrange 切断は $B$ の開被覆 $(U_i)_{i\in I}$ 
とその上の関数
Lagrange 切断 $s$ が、
零切断 $s_0$ と横断的に交わる時、
$p\in s_0\cap s$の開近傍上の関数$f_p\colon U_p\to \mathbb{R}$
で、$df_p=s|_{U_p}$ かつ $p$で孤立特異点を持つもので 
$\opn{HF}^{\bullet}(\mathscr{L}_{s_0},\mathscr{L}_s)$ を
定義する複体は次のようになる\cite{MR1882331}。
\begin{align}
\opn{CF}^{\bullet}(\mathscr{L}_{s_0},
\mathscr{L}_s)=\bigoplus_{p\in \check{f}_{B}(\opn{Im}(s_0)\cap \opn{Im}(s))}
\Lambda^{\mathbb{C}}_{\mathrm{nov}}[-\opn{ind}_{\mathrm{Morse}}(f_p,p)]
\end{align}
ここで
$\opn{ind}_{\mathrm{Morse}}(f_p,p)$
とは、$f_p$ の $p$ での Morse指数であり、
$\Lambda^{\mathbb{C}}_{\mathrm{nov}}$ は
$\mathbb{C}$ 上の Novikov 体である。
ここで注目するべき点は、
$\opn{HF}^{\bullet}(\mathscr{L}_{s_0},\mathscr{L}_{s})$
は SYZ変換によって層コホモロジー
$H^{\bullet}(\mathcal{X}(B);
\Phi_{\opn{SYZ}}(\mathscr{L}_{s}))$
と同型であり\cite{MR4301560}、
$\opn{CF}^{\bullet}(\mathscr{L}_{s_0},
\mathscr{L}_s)$は次数つきベクトル空間としては、
トロピカル幾何の言葉のみで表現されていることの二点である。

これらを踏まえた上で博士論文提出者は、SYZ 予想のトイモデルとして
次のメタ予想を提示する。

\begin{expectation} \label{conjecture-mirror-tropical-rr}

$X$ をコンパクトなトロピカル多様体とし、
$\mathcal{L}$ を $X$ 上のトロピカル直線束とする。
ここで、$X$ には `Todd 類' $\opn{td}(X)$ が
トロピカルコホモロジー群 $H^{\bullet,*}(X;\mathbb{Z})$
の元として定義されていると仮定する。
このとき、$\mathcal{L}$ を表現するしかるべき条件を満たした
$C^{\infty}$ Cartier 因子 $s$ が存在し、$s$ が定義する次数つき
$\mathbb{Z}$-加群 $\opn{LMD}^{\bullet}(X;s)$ に対して、
$s$ の取り方によらず
次の等式が成り立つ:
\begin{align}
\chi(\opn{LMD}^{\bullet}(X;s))=\int_{X}\opn{ch}(\mathcal{L})\opn{td}(X).
\end{align}
ここで、$\int_X$ はトレース写像であり、
$\mathrm{ch}(\mathcal{L})$ は、$\mathcal{L}$ の全 Chern 類である。
\end{expectation}
$\opn{LMD}^{\bullet}(X;s)$ は次節で述べるが、
コンパクト整アフィン多様体 $B$と横断的に交わる Lagrange 切断$s$
に対しては、次数つきベクトル空間として
$\opn{LMD}^{\bullet}(B;s)\otimes_{\mathbb{Z}}
\Lambda^{\mathbb{C}}_{\mathrm{nov}}=
\opn{CF}^{\bullet}(\mathscr{L}_{s_0},
\mathscr{L}_s)$ となる。

上述の $X$ 上の滑らかな因子 $s$ のしかるべき条件とは、本論文では次のような条件である。

\begin{condition} \label{condition-good}
(i) $X$ の座標系がすべて $\mathbb{R}^{n}$ 
の凸多面体の開集合として取れるトロピカル多様体である。
(ii) $s$ が permissible である。
(iii) $\sharp (\opn{Im}(s_0)\cap \opn{Im}(s))<\infty$。
(iv) $\opn{LMD}^{\bullet}(X;s)$ が有限生成である。
\end{condition}
なお条件(iv)は将来的に不要となると期待している。
いくつか用語の説明を後回しにして、
以上を踏まえた上での本論文の主結果(の弱い版)
を述べると以下のようになる。

\begin{theorem} \label{theorem-main}
$C$ (resp. $B$) をコンパクトなトロピカル曲線 (resp. Hesse 計量を持つ
整アフィン多様体) とし、$\mathcal{L}$ をその上の直線束
$s$ を $C$ (resp. $B$) 上の
\cref{condition-good} を満たす滑らかな因子 $s$ 
とする。このとき、次が成り立つ。
\begin{align}
\chi(\opn{LMD}^{\bullet}(C;s))=\int_C c_1(\mathcal{L})+
\chi_{\opn{top}}(C), \quad
\chi(\opn{LMD}^{\bullet}(B;s))=\int_B \frac{c_1(\mathcal{L})^n}{n!}
\end{align}
\end{theorem}
トロピカル曲線の定理\ref{theorem-main} の右辺は
(\ref{equation-tropical-rr}) の右辺と同じである。
一般にトロピカル多様体の Todd 類 は定義されていないが、
トロピカル多様体の Chern 類が
満たすべき条件から $\opn{td}(B)=1$ であるべきであることが
わかるため、定理\ref{theorem-main} は
\cref{conjecture-mirror-tropical-rr}
への肯定的な部分解答であるといえる。
この二つの結果はSYZ変換を介して \cite{MR4301560} や 
\cite{auroux2022lagrangian} の特殊例ともみなせるが、
本提出論文の証明はトロピカル幾何学の言葉のみで完結しており、
左辺に関しても上述のものとは異なるものである。

\beforesection

\section{局所 Morse データと証明の概略}

\aftersection

この節では、\cref{condition-good} (i) を満たしている
有理凸多面体空間 $S$ とその上の滑らかな因子
$s\in H^{0}(S;\mathcal{A}_{S}^{0,0}/\mathcal{O}^{\times}_S)$
に対する permissiblity 条件と 
$\opn{LMD}^{\bullet}(S;s)$ について説明する。
ここで $\mathcal{A}^{0,0}_S$ は $S$ 上の $(0,0)$-superform
のなす層のことである\cite{MR3903579}。まず、$s$は局所的には、$X$の開集合上の 
$(0,0)$ superform $f\colon U\to \mathbb{R}$
によって表現されており、各点 $x$ において
全微分ベクトル $df(x)\in(\mathcal{F}_{S}^{1})_x$ 
が定義される。$\mathcal{F}_{S}^{1}$ とは、定数層
$\mathbb{R}_S$ による
商層 $\mathcal{O}^{\times}_{S}/\mathbb{R}_{S}$ 
の $\mathbb{R}$ 係数拡大であり、
$(\mathcal{F}_{S}^{1})_x$の双対空間
$T_x S$は、$S$ の $x$ の局所的な錐
$\opn{LC}_x S$ の張るベクトル空間と同一視される。
$\phi \colon \opn{LC}_x S\to T_x S$ を上述の同一視の下
による閉埋め込みとすると
$\opn{SS}_x (S)$を
$\opn{SS}_0(\phi_* \phi^{-1}\mathbb{Z}_{T_x S})$
と定義する。このとき、
$\opn{SS}_0(\phi_* \phi^{-1}\mathbb{Z}_{T_x S})$は、
$T_x X$を微分多様体とみなしたときの、
$\phi_* \phi^{-1}\mathbb{Z}_{T_x S}$ のマイクロ台と
原点での$T_x X$の余接ベクトル空間との交叉である。
このとき、
$f\colon U \to \mathbb{R}$ が $x$ にて permissible
とは、次の条件を満たすことである。
\begin{align}
df(x)\notin\opn{span}_{\mathbb{R}}(\opn{SS}_x(S))+
(\mathcal{F}_{\mathbb{Z},S}^{1})_x 
\setminus (\mathcal{F}_{\mathbb{Z},S}^{1})_x
\end{align}

$S$ には、各点に対してトーラス
$\check{X}_0(S)_x\deq ((\mathcal{F}^{1}_{\mathbb{Z},S})_x/
\opn{span}_{\mathbb{R}}(\opn{SS}_x(S))\cap 
(\mathcal{F}^{1}_{\mathbb{Z},S})_x)\otimes_{\mathbb{Z}}
\mathbb{R}/\mathbb{Z}$ が定義され、
permissible な$C^{\infty}$級因子 $s$ に対して写像
$\check{s}\colon S\to \check{X}_0(S)\deq 
\bigcup_{x\in S} \check{X}_0(S)_x$ が定義される。 
そして標準的な射影 $\check{f}_{S}\colon \check{X}_0(S)\to S$
によって、自明な因子 $s_0$ と $s$ の交点は
$s_0\cap s\deq \check{f}_{S}(\opn{Im}(\check{s}_0)\cap 
\opn{Im}(\check{s}))$ と定義される。
この、$s_0\cap s$ が有限集合のとき、
$s$ の局所 Morse データは層の局所 Morse データを用いることで
次のように定義される。
\begin{align} \label{equation-local-morse-data}
\opn{LMD}^{\bullet}(S;s)\deq 
\bigoplus_{p\in s_0\cap s} 
(R^{\bullet}_{\{x\in U_p\mid f_p(x)\geq f_p(p)\}}(\mathbb{Z}_{U_p}))_p.
\end{align}
ここで、$f_p\colon U_p\to \mathbb{R}$ とは、
$p$ の十分小さな近傍 $U_p$ 上の $(0,0)$-superform で
$s|_{U_p}$ が $f_p$ の定義する滑らかな主因子と一致し、
$p$ にのみ特異点を持つものである。 
\cref{equation-local-morse-data}の右辺の直和成分は
局所 Morse データもしくは、micro-stalk と呼ばれるものの
コホモロジー群であり、超局所層理論において基本的なものである
\cite{MR2031639,MR1299726}。

コンパクトトロピカル曲線における、定理\ref{theorem-main}の証明は、
各閉区間に対して定理\ref{theorem-main}を証明し、
それらを張り合わせることで証明する。
Hesse 計量をもつコンパクト整アフィン多様体については、
Hesse 計量をもつコンパクト整アフィン多様体は Cheng--Yau
の結果\cite{MR714338}より、トロピカルトーラスの有限不分岐被覆によって
得られることとトロピカルコホモロジーとトロピカル Borel--Moore
ホモロジーとのカップ積に関する射影公式
\cite{gross2019sheaftheoretic}を利用することで
トロピカルトーラスの場合に帰着させることで示した。
また本論文では、有理多面体空間 $S$ 上の滑らかな因子 $s$ と
有理多面体空間 $S'$上の滑らかな因子$s'$が\cref{condition-good}
を満たしかつコホモロジー的な Milnor 条件を満たすとき、
K\"unneth 型の定理が成り立つことを層における
Thom--Sebastiani型の定理\cite{MR2031639}を
用いることで示している。

\aftersection

{\small
\bibliography{lattice_points_surface}
\bibliographystyle{halpha}
}
\end{document}