\documentclass[uplatex,dvipdfmx,12pt]{jsarticle}

%%%% packages
\usepackage[utf8]{inputenc}
\usepackage[dvipdfmx]{graphicx,color}%for images
\usepackage{bm}%fonts
\usepackage{tikz-cd}%
\usetikzlibrary{cd}
\usetikzlibrary{calc}
\usepackage{amsmath,amsthm,amstext,amsfonts,amsbsy}% ほぼ必須
\usepackage{amssymb}
\usepackage{latexsym}% ほぼ必須
\usepackage{makecell}%表のセル内で改行するためのパッケージ
\usepackage{algpseudocode,algorithm}%疑似コード用
%\usepackage{todonotes}%comments
\usepackage[top=0.8in,left=0.8in,right=0.8in,
bottom=1.1in]{geometry}
\usepackage{layout}
\usepackage[T1]{fontenc}%font encoding
\usepackage{physics}
\usepackage{braket}%after physics
\usepackage{mathtools,thmtools}
\usepackage{imakeidx}%before hyperref for pagebackref
\usepackage[dvipdfmx]{hyperref}
\usepackage[capitalize]{cleveref}
\hypersetup{
     colorlinks = true,
     citecolor  = blue,
     linkcolor  = blue, 
     urlcolor   = blue, 
}
\usepackage{mathrsfs}
\usepackage{upgreek}
\numberwithin{equation}{section}

\usepackage{titlesec}

\titleformat*{\section}{\large\bfseries}

%%%% theoremstyle

\theoremstyle{definition}
\newtheorem{theorem}{定理}[section]
\newtheorem*{theorem*}{定理}
\newtheorem{definition}[theorem]{Definition}
\newtheorem{definition*}{Definition}
\newtheorem{example}[theorem]{Example}
\newtheorem*{example*}{Example}
\newtheorem{proposition}[theorem]{Proposition}
\newtheorem*{proposition*}{Proposition}
\newtheorem{Note}[theorem]{Note}
\newtheorem*{Note*}{Note}
\newtheorem{Ntc}[theorem]{Notice}
\newtheorem*{Ntc*}{Notice}
\newtheorem{lemma}[theorem]{Lemma}
\newtheorem*{lemma*}{Lemma}
\newtheorem{Fact}[theorem]{Fact}
\newtheorem*{Fact*}{Fact}
\newtheorem{question}[theorem]{Question}
\newtheorem*{question*}{Question}
\newtheorem{conjecture}[theorem]{予想}
\newtheorem*{conjecture*}{予想}
\newtheorem{Rule}[theorem]{Rule}
\newtheorem*{Rule*}{Rule}
\newtheorem{notation}[theorem]{Notation}
\newtheorem*{Not*}{Notation}
\newtheorem{corollary}[theorem]{Corollary}
\newtheorem*{corollary*}{Corollary}
\newtheorem{remark}[theorem]{Remark}
\newtheorem*{Rmk*}{Remark}
\newtheorem{condition}[theorem]{条件}
\newtheorem*{condition*}{条件}
\newtheorem{Conv}[theorem]{Convention}
\newtheorem*{Conv*}{Convention}
\newtheorem{observation}[theorem]{Observation}
\newtheorem*{observation*}{Observation}
\newtheorem{expectation}[theorem]{メタ予想}
\newtheorem{fact}[theorem]{事実}

% newcommand
\newcommand{\deq}{\coloneqq}
\newcommand{\opn}[1]{\operatorname{#1}}
\newcommand{\catn}[1]{\mathbf{#1}}
\newcommand{\beforesection}{\vspace{-15pt}}
\newcommand{\aftersection}{\vspace{-8pt}}

\DeclareMathOperator{\tform}{\Omega}

\title{論文内容の要旨
}
%\author[Y. Tsutsui]{Yuki Tsutsui}
% \address{Graduate School of Mathematical Sciences,
% The University of Tokyo, 3-8-1 Komaba, Meguro-Ku,
% Tokyo, 153-8914, Japan}
% \email{tyuki@ms.u-tokyo.ac.jp}
\date{}

%layout

%package

\usepackage{lineno}

\newcommand*\patchAmsMathEnvironmentForLineno[1]{
  \expandafter\let\csname old#1\expandafter\endcsname\csname #1\endcsname
  \expandafter\let\csname oldend#1\expandafter\endcsname\csname end#1\endcsname
  \renewenvironment{#1}
     {\linenomath\csname old#1\endcsname}
     {\csname oldend#1\endcsname\endlinenomath}}
\newcommand*\patchBothAmsMathEnvironmentsForLineno[1]{
  \patchAmsMathEnvironmentForLineno{#1}
  \patchAmsMathEnvironmentForLineno{#1*}}
\AtBeginDocument{
\patchBothAmsMathEnvironmentsForLineno{equation}
\patchBothAmsMathEnvironmentsForLineno{align}
\patchBothAmsMathEnvironmentsForLineno{flalign}
\patchBothAmsMathEnvironmentsForLineno{alignat}
\patchBothAmsMathEnvironmentsForLineno{gather}
\patchBothAmsMathEnvironmentsForLineno{multline}
}

\usepackage{framed}
\setlength{\OuterFrameSep}{-4pt}
\setlength{\FrameSep}{3pt}

\begin{document}
\setlength{\baselineskip}{-5pt}
\setlength{\parskip}{2pt}

%\linenumbers

\maketitle
{\large
\noindent
論文題目:

Graded modules associated 
with permissible $C^{\infty}$-divisors on tropical manifolds

(トロピカル多様体上の可容$C^{\infty}$因子に付随した次数付き加群)
}

\vspace{5pt}

\noindent
{\large
氏名: 筒井 勇樹
}

\vspace{10pt}

\setlength{\baselineskip}{4pt}
\setlength{\parskip}{3pt}

本論文の目的は
(1)コンパクトトロピカル多様体上の直線束を表現する
可容$C^{\infty}$因子に対して次数付き加群を
定義することと
(2)その定式化から自然と提起される Riemann--Roch 型の定理の
新たな類似をトロピカル曲線と Hesse 計量をもつ整アフィン多様体に対して
証明することである。なおこの類似は Gathmann--Kerber
によるトロピカル Riemann--Roch の定理とは異なる。

\beforesection

\section{Riemann--Rochの定理のトロピカル幾何的類似の研究の背景}

\aftersection

トロピカル幾何とは、ある条件を満たす多面複体の台空間 $S$ とその上の
局所的に整数係数アフィンな関数のなす層 
$\mathcal{O}_S^{\times}$
の組を代数多様体の類似物とみなして研究する分野である
\cite{mikhalkinTropicalEigenwaveIntermediate2014a,
gross2019sheaftheoretic}。
そのような空間をトロピカル多様体と呼ぶ。
一次元トロピカル多様体をトロピカル曲線と呼び、
有限距離グラフは標準的なトロピカル曲線の構造を持つ。
トロピカル多様体であり、境界なし微分多様体の構造を自然にもつ空間を
整アフィン多様体と呼ぶ。
Hesse 計量は整アフィン多様体における
K\"ahler 計量の類似である。

コンパクトトロピカル曲線 $C$ 上の Cartier 因子 
$D\in \opn{CDiv}(C)$
に関する Riemann--Roch の定理は
Gathmann--Kerber
\cite{gathmannRiemannRochTheoremTropical2008a} によって 
Baker--Norine による
グラフの Riemann--Roch の定理の一般化として
次のような等式として証明された。
\begin{align} \label{equation-tropical-rr}
r(D)-r(K_C-D)=\opn{deg}(D)+\chi_{\mathrm{top}}(C)
\end{align}
ここで、$r(D)$ とは、$D$ の因子の線形系の階数であり、
$K_C$ は$C$ の標準因子、$\chi_{\mathrm{top}}(C)$ は
$C$ の位相的 Euler 数である。
この等式を高次元に一般化することは自然な考えであるが、
両辺ともに技術的な困難が存在する。
% 代数幾何における
% Hirzebruch--Riemann--Roch の定理では、
% 左辺は因子に付随する直線束の層コホモロジーの
% Euler 数に、右辺は 直線束の Chern 類と Todd 類の
% カップ積で表現されるが、トロピカル幾何においては
% 直線束の層コホモロジーと Todd 類の類似物は一般には
% 定義されていない。
特に左辺は、
トロピカル直線束は Abel 群のなす層ではないという
根本的な問題が原因である。
% \textcolor{gray}{
% 一方で、数多くの研究結果から
% トロピカル多様体の構造層の Euler 数の類似は
% トロピカル多様体上の定数層の Euler 数であると
% 考えられている。
% }

\beforesection

\section{SYZ予想および本研究の主結果}

\aftersection

本研究では、Strominger--Yau--Zaslow 予想 (SYZ予想)
のアイデアと超局所層理論\cite{MR1299726}を
利用することで、
コンパクトトロピカル多様体上の直線束を表現する 
可容$C^{\infty}$因子に対して
Floer 複体もしくは Morse 複体の類似の
次数付き加群を定義し、その次数つき加群の 
Euler 数を 直線束の Euler 数のトロピカル類似とする。

SYZ 予想\cite{stromingerMirrorSymmetryTduality1996}
とは、Calabi--Yau 多様体 $\check{X}$ の
ミラー Calabi--Yau 多様体 $\mathcal{X}$ が、
ある位相多様体 $B$ への(特異ファイバーを許す)
特殊 Lagrange トーラスのファイブレーション
$\check{f}_{B}\colon \check{X}\to B$ の
双対 $f_{B}\colon 
\mathcal{X}\to B$
として得られるという弦理論に由来する数理物理の
予想であり、この予想の数学的定式化は、
$\check{X}$の導来深谷圏 $\mathrm{fuk}(\check{X})$ から
 $\mathcal{X}$ 上の連接層の導来圏$\mathrm{coh}(\mathcal{X})$
への擬同型関手$\Phi_{\mathrm{SYZ}}$ を誘導すると
期待されている。$\Phi_{\mathrm{SYZ}}$を
SYZ変換と呼ぶ。
% $\Phi_{\opn{SYZ}}$が
% 存在するならば、Floer コホモロジーには
% 次の同型
% \begin{align}
% \opn{HF}^{\bullet}(\mathscr{L}_1,\mathscr{L}_2)
% \simeq \opn{Ext}^{\bullet}_{\mathcal{O}_{\mathcal{X}}}(\Phi_{\opn{SYZ}}(\mathscr{L}_1),
% \Phi_{\opn{SYZ}}(\mathscr{L}_2))
% \end{align}
% が存在する。
ここで
$\mathscr{L}_1$, $\mathscr{L}_2$ は、$\catn{fuk}(\check{X})$
の対象であり、コンパクト Lagrange 部分多様体といくつかの追加
データによって構成されている。
% $[L_1]$, $[L_2]$ をそれぞれ
% $\mathscr{L}_1$, $\mathscr{L}_2$ の Lagrange 部分多様体
% のホモロジー類だとすると、定義より
% \begin{align}
% \chi(\opn{HF}^{\bullet}(\mathscr{L}_1,\mathscr{L}_2))=
% (-1)^{\frac{n(n-1)}{2}} \sharp ([L_1]\cap [L_2])
% \end{align}
% であることが導かれる。
元々のSYZ予想はCalabi--Yau 多様体に対する予想だが、
Mumford 曲線に一般化したもの\cite{auroux2022lagrangian}
も考察されている。

$B$ を
$\pi_2(B)=0$となる
コンパクト整アフィン多様体とする。このとき、$B$ の
構造から誘導される特殊Lagrangeトーラス束 
$\check{X}(B)\to B$ と 
双対トーラス束 $\mathcal{X}(B)\to B$ が定義される。
この$\check{X}(B)$と$\mathcal{X}(B)$のSYZ変換は
\cite{MR1882331,MR4301560}などで深く研究されている。

\begin{fact}
\hspace{-8pt} (i) $\mathbb{Z}$係数の傾きをもつ局所アフィン
な関数のなす層
$\mathcal{O}_B^{\times}$から$C^{\infty}$級関数のなす層
$\mathcal{C}^{\infty}_B$への単射がトロピカルPicard群への全射
$H^{0}(B;\mathcal{C}^{\infty}_B/\mathcal{O}_B^{\times})
\to H^{1}(B;\mathcal{O}_B^{\times})$ を導く。

\hspace{-10pt} (ii) 任意の元 $s\in 
H^{0}(B;\mathcal{C}^{\infty}_B/\mathcal{O}_B^{\times})$ 
はLagrange切断$\mathscr{L}_s$と呼ばれる、
$\catn{fuk}(\check{X}(B))$ の対象とみなせる、Lagrange
部分多様体と対応し、
$\mathscr{L}_s$のSYZ変換$\mathcal{L}_s$は$\mathcal{X}(B)$上の
直線束である。

\hspace{-12pt} (iii) SYZ変換は、Floerコホモロジー
$\opn{HF}^{\bullet}(\mathscr{L}_s,\mathscr{L}_{s'})$
と$\opn{Ext}^{\bullet}
(\mathcal{L}_s,\mathcal{L}_{s'})$ との同型を誘導する。

\hspace{-12pt} (iv) 
$\mathscr{L}_s$と$\mathscr{L}_{s'}$が横断的に交わるなら、
$\opn{HF}^{\bullet}(\mathscr{L}_s,\mathscr{L}_{s'})$を
定義する複体$\opn{CF}^{\bullet}(\mathscr{L}_s,\mathscr{L}_{s'})$
の次数付き加群成分は、$B,s,s'$のみで記述できる。
\end{fact}
% $B$ 上のトロピカル直線束は、$H^{1}(B;\mathcal{O}^{\times}_B)$
% によってパラメトライズされており、滑らかな関数のなす層 
% $\mathcal{C}^{\infty}(B)$ への単射 
% $\mathcal{O}_{B}^{\times}\to \mathcal{C}^{\infty}(B)$ が誘導
% する $H^{0}(B;\mathcal{C}^{\infty}(B)/\mathcal{O}_{B}^{\times})$
% から $H^{1}(B;\mathcal{O}^{\times}_B)$ への全射を持つ。
% $H^{0}(B;\mathcal{C}^{\infty}(B)/\mathcal{O}_{B}^{\times})$
% の元は、整アフィン多様体上のトロピカル Cartier 因子の
% 類似であり、本論文では$C^{\infty}$級
% Cartier データと
% 読ぶ。このとき、
% \begin{align} \label{equation-lagrangian-section}
% H^{0}(B;\mathcal{C}^{\infty}(B)/\mathcal{O}_{B}^{\times})
% \simeq \{s\colon B\to \check{X}(B)\mid 
% \check{f}_{B}\circ s=\opn{id}_B, \,
% \opn{Im}(s) \text{ が Lagrange 部分多様体} \}
% \end{align}
% が成り立つ。(\ref{equation-lagrangian-section}) の
% 右辺に属する元を Lagrange 切断と呼び、
% 自然と $\catn{fuk}(\check{X}(B))$ の対象
% $\mathscr{L}_s$ とみなせ、
% SYZ 変換によって$\mathcal{X}(B)$ 上の直線束へと
% 変換される。同型 (\ref{equation-lagrangian-section})
% より Lagrange 切断は $B$ の開被覆 $(U_i)_{i\in I}$ 
% とその上の関数
% Lagrange 切断 $s$ が、
% 零切断 $s_0$ と横断的に交わる時、
% $p\in s_0\cap s$の開近傍上の関数$f_p\colon U_p\to \mathbb{R}$
% で、$df_p=s|_{U_p}$ かつ $p$で孤立特異点を持つもので 
% $\opn{HF}^{\bullet}(\mathscr{L}_{s_0},\mathscr{L}_s)$ を
% 定義する複体は次のようになる\cite{MR1882331}。
% \begin{align}
% \opn{CF}^{\bullet}(\mathscr{L}_{s_0},
% \mathscr{L}_s)=\bigoplus_{p\in \check{f}_{B}(\opn{Im}(s_0)\cap \opn{Im}(s))}
% \Lambda^{\mathbb{C}}_{\mathrm{nov}}[-\opn{ind}_{\mathrm{Morse}}(f_p,p)]
% \end{align}
% ここで
% $\opn{ind}_{\mathrm{Morse}}(f_p,p)$
% とは、$f_p$ の $p$ での Morse指数であり、
% $\Lambda^{\mathbb{C}}_{\mathrm{nov}}$ は
% $\mathbb{C}$ 上の Novikov 体である。
% ここで注目するべき点は、
% $\opn{HF}^{\bullet}(\mathscr{L}_{s_0},\mathscr{L}_{s})$
% は SYZ変換によって層コホモロジー
% $H^{\bullet}(\mathcal{X}(B);
% \Phi_{\opn{SYZ}}(\mathscr{L}_{s}))$
% と同型であり\cite{MR4301560}、
% $\opn{CF}^{\bullet}(\mathscr{L}_{s_0},
% \mathscr{L}_s)$は次数つきベクトル空間としては、
% トロピカル幾何の言葉のみで表現されていることの二点である。

上の事実を踏まえ、次の予想を提示する。

\begin{conjecture} \label{conjecture-mirror-tropical-rr}
任意のコンパクトなトロピカル多様体$S$に対して、
トロピカルコホモロジー群 $H^{\bullet,\bullet}(S;\mathbb{Z})$
の元$\opn{td}(S)$が存在して、
$S$上の任意の
$C^{\infty}$因子$s$で、
次節で述べる可容条件(\cref{condition-good})を満たすものに対し、
次の等式が成り立つ:
\begin{align}
\chi(\opn{LMD}^{\bullet}(S;s))=
\int_{S}\opn{ch}([s])\opn{td}(S).
\end{align}
ここで、$\int_S$ はトレース写像であり、
$\mathrm{ch}([s])$は$s$の因子類$[s]\in \opn{Pic}(S)$の 
Chern指標である。
\end{conjecture}
$C^{\infty}$因子と
$\opn{LMD}^{\bullet}(S;s)$は次節で述べるが、
整アフィン多様体$B$上の$C^{\infty}$因子は、
$H^{0}(B;\mathcal{C}_{B}^{\infty}/\mathcal{O}^{\times}_B)$
の元のことであり、
コンパクト整アフィン多様体 $B$と
$\mathscr{L}_s$が零切断と横断的に交わる$s$
の組に対しては、複素係数Novikov体 
$\Lambda^{\mathbb{C}}_{\mathrm{nov}}$
上の次数つき加群として
$\opn{LMD}^{\bullet}(B;s)\otimes_{\mathbb{Z}}
\Lambda^{\mathbb{C}}_{\mathrm{nov}}\simeq
\opn{CF}^{\bullet}(\mathscr{L}_{s_0},
\mathscr{L}_s)$ となる。
% なお条件(iv)は将来的に不要となると期待している。
任意の元$\mathcal{L}\in\opn{Pic}(S)$に対して、
$[s]=\mathcal{L}$となる可容$C^{\infty}$因子
$s$が
存在するか否かは一般には非自明であるが、豊富に存在すると期待
できる。実際に$S$がコンパクトトロピカル曲線もしくは整アフィン多様体
の場合は豊富に存在する。
本論文の主結果は、\cref{conjecture-mirror-tropical-rr}
への肯定的な部分解答であり、
用語や条件の説明を後回しにすると、
以下のようになる。

\begin{theorem} \label{theorem-main}
$C$ (resp. $B$) をコンパクトなトロピカル曲線 (resp. Hesse 計量を持つ
$n$次元整アフィン多様体) とし、
$s$ を $C$ (resp. $B$) 上の
\cref{condition-good} を満たす$C^{\infty}$因子 
とする。このとき、次が成り立つ。
\begin{align}
\chi(\opn{LMD}^{\bullet}(C;s))=\int_C c_1([s])+
\chi_{\opn{top}}(C), \quad
\chi(\opn{LMD}^{\bullet}(B;s))=\int_B \frac{c_1([s])^n}{n!}
\end{align}
\end{theorem}
% トロピカル曲線の定理\ref{theorem-main} の右辺は
% (\ref{equation-tropical-rr}) の右辺と同じである。
% 一般にトロピカル多様体の Todd 類 は未定義だが、
% トロピカル多様体の Chern 類が
% 満たすべき条件から $\opn{td}(B)=1$ であるべきだと
% わかるため、定理\ref{theorem-main} は
% \cref{conjecture-mirror-tropical-rr}
% への肯定的な部分解答であるといえる。
% \textcolor{gray}{
% この結果はSYZ変換を介して \cite{MR4301560} や 
% \cite{auroux2022lagrangian} の特殊例ともみなせるが、
% 本提出論文の証明はトロピカル幾何学の言葉のみで完結しており、
% 左辺に関しても上述のものとは異なるものである。}

\beforesection

\section{局所 Morse データと証明の概略}

\aftersection

この節では、有理多面空間$S$とその上の$C^{\infty}$因子
$s$の可容条件(\cref{condition-good})と 
$\opn{LMD}^{\bullet}(S;s)$ について説明する。
まず、$S$の各点$x$には、余接ベクトル空間
$(\tform^{1}_{S})_x\deq
(\mathcal{O}_S^{\times}/\mathbb{R}_S)_x
\otimes_{\mathbb{Z}}\mathbb{R}$
と、その中の整数ベクトルのなす格子
$(\tform^{1}_{\mathbb{Z},S})_x$が存在する。
接ベクトル空間
$T_x S\deq(\tform^{1}_{S})_x^{\vee}$の中には、
local coneと呼ばれる
$T_xS$を張る有理多面集合
$\opn{LC}_x S$が存在する。
$\phi \colon \opn{LC}_x S\to T_x S$を埋め込み写像とすると、
$T^{*}(T_xS)$の部分集合である層のマイクロ台
$\opn{SS}(\phi_!\phi^{-1}\mathbb{Z}_{T_xS})$が定義される。
このとき、原点での余接ベクトル空間$T^{*}_0(T_xS)$が
$(\tform_S^{1})_x$と自然に同型であることに注意することで、
$(\tform_S^{1})_x$の部分集合
$\opn{SS}(S)_x\deq \opn{SS}(\phi_!\phi^{-1}\mathbb{Z}_{T_xS})
\cap (\tform_S^{1})_x$が定義される。

$S$上の$C^{\infty}$因子$s$は、$S$上の
weakly-smoothな関数のなす層$\mathcal{A}^{\mathrm{weak}}_S$
によって定義されるコホモロジー
$H^{0}(S;\mathcal{A}^{\mathrm{weak}}_S/\mathcal{O}^{\times}_S)$
の元である。
なお、$S$がboundarylessのとき、つまり
$S$ の座標系がすべてある$\mathbb{R}^{n}$ 
の有理凸多面体の有限和の開集合として取れるような$S$のとき、
$\mathcal{A}^{\mathrm{weak}}_S$は$S$上の(0,0)-superformのなす
層$\mathcal{A}^{0,0}_S$ \cite{MR3903579}と一致する。
$s$は局所的には、
$S$の開集合上の 
weakly-smoothな関数 $f\colon U\to \mathbb{R}$
の主因子によって表現されており、各点 $x$ において
全微分ベクトル $df(x)\in(\tform_{S}^{1})_x$ 
が定義される。
このとき、
$f\colon U \to \mathbb{R}$ が $x$ にて
前可容(prepermissible)
とは、次の条件を満たすことを指す。
\begin{align}
df(x)\notin\opn{span}_{\mathbb{R}}(\opn{SS}(S)_x)+
(\tform_{\mathbb{Z},S}^{1})_x 
\setminus (\tform_{\mathbb{Z},S}^{1})_x
\end{align}
$C^{\infty}$因子$s$が前可容とは、
$s$が前可容なweakly-smoothな関数のなす
局所データ$\{(U_i,f_i)\}_{i\in I}$
をもつことをいう。

$S$上の各点$x$に対してトーラス
$\check{X}_0(S)_x\deq ((\tform^{1}_{\mathbb{Z},S})_x/
\opn{span}_{\mathbb{R}}(\opn{SS}_x(S))\cap 
(\tform^{1}_{\mathbb{Z},S})_x)\otimes_{\mathbb{Z}}
\mathbb{R}/\mathbb{Z}$ が定義され、
前可容な$C^{\infty}$因子 $s$ に対して写像
$\check{s}\colon S\to \check{X}_0(S)\deq 
\bigcup_{x\in S} \check{X}_0(S)_x$ が定義される。 
標準的な射影 $\check{f}_{S}\colon \check{X}_0(S)\to S$
によって、自明な因子 $s_0$ と $s$ の交点は
$s_0\cap s\deq \check{f}_{S}(\opn{Im}(\check{s}_0)\cap 
\opn{Im}(\check{s}))$ と定義される。
この$s_0\cap s$ が有限集合のとき、
$s$ の局所 Morse データは層の局所 Morse データのコホモロジー
を用いることで次のように定義される。
\begin{align} \label{equation-local-morse-data}
\opn{LMD}^{\bullet}(S;s)\deq 
\bigoplus_{p\in s_0\cap s} 
(R^{\bullet}_{\{x\in U_p\mid f_p(x)\geq f_p(p)\}}(\mathbb{Z}_{U_p}))_p
\end{align}
ここで$f_p\colon U_p\to \mathbb{R}$は、
$p$ の十分小さな近傍 $U_p$ 上の weakly-smoothな関数で、
$s|_{U_p}$ が $f_p$ の定義する$C^{\infty}$因子と一致し、
$p$ にのみ臨界点を持つものである。
% \textcolor{gray}{ 
% \cref{equation-local-morse-data}の右辺の直和成分は
% 局所 Morse データもしくは、micro-stalk と呼ばれるものの
% コホモロジーであり、超局所層理論において基本的なものである
% \cite{MR2031639,MR1299726}。
% }

$C^{\infty}$因子$s$が
可容である(permissible)とは、
次の条件を満たす事を指す。

\begin{condition} \label{condition-good}
(i) $s$ が前可容である。
(ii) $\sharp (s_0\cap s)<\infty$。
(iii) $\opn{LMD}^{\bullet}(S;s)$ が有限生成。
\end{condition}


コンパクトなトロピカル曲線に対する定理\ref{theorem-main}の証明は、
各閉区間に対して定理\ref{theorem-main}を証明し、
それらを張り合わせることで証明する。
Hesse 計量をもつコンパクト整アフィン多様体については、
Hesse 計量をもつコンパクト整アフィン多様体は Cheng--Yau
の結果\cite{MR714338}より、トロピカルトーラスの有限不分岐被覆によって
得られることとトロピカルコホモロジーとトロピカル Borel--Moore
ホモロジーの射影公式
\cite{gross2019sheaftheoretic}を利用することで
示した。
また本論文では、\cref{condition-good}を少し強めた
$C^{\infty}$因子に対して
K\"unneth 型の定理が成り立つことを層における
Thom--Sebastiani型の定理\cite{MR2031639}を
用いることで示している。

\aftersection

{\small
\bibliography{lattice_points_surface}
\bibliographystyle{halpha}
}
\end{document}