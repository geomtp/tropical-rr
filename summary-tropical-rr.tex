\documentclass[a4paper,dvipdfmx,reqno,12pt]{amsart}
\synctex=1
%
%%%% packages
\usepackage[utf8]{inputenc}
\usepackage[dvipdfmx]{graphicx,color}%for images
\usepackage{bm}%fonts
\usepackage{tikz-cd}%
\usetikzlibrary{cd}
\usetikzlibrary{calc}
\usepackage{amsmath,amsthm,amstext,amsfonts,amsbsy}% ほぼ必須
\usepackage{amssymb}
\usepackage{latexsym}% ほぼ必須
\usepackage{makecell}%表のセル内で改行するためのパッケージ
\usepackage{algpseudocode,algorithm}%疑似コード用
\usepackage{todonotes}%comments
\usepackage[margin=0.8in]{geometry}
\usepackage{layout}
\usepackage[T1]{fontenc}%font encoding
\usepackage{physics}
\usepackage{braket}%after physics
\usepackage{mathtools,thmtools}
\usepackage{imakeidx}%before hyperref for pagebackref
\usepackage[pagebackref,dvipdfmx]{hyperref}
\usepackage[capitalize]{cleveref}
\hypersetup{
     colorlinks = true,
     citecolor  = blue,
     linkcolor  = blue, 
     urlcolor   = blue, 
}
%\usepackage{pxjahyper}%for hyperref in Japanese
\usepackage{bookmark}
\usepackage{dynkin-diagrams}

%%%% imakeidx
\makeindex
\makeindex[name=not, title=Index, columns=2]
\makeindex[name=sym, title=Symbol, columns=3]
\makeindex[name=ref, title=Ref, columns=3]

\newcommand{\ind}[2]{\emph{#1}\index{1{#2}@{#1}}}
\newcommand{\indset}[3]{$#1 \deq #2 $ \index{0{#3}@$#1$} }
\newcommand{\indse}[2]{{$#1$}\index{0{#2}@{$#1$}}}

%%%%
\usepackage{pgf,tikz,pgfplots}
\pgfplotsset{compat=1.15}
\usetikzlibrary{arrows}



%%%%


%%%% theoremstyle

\theoremstyle{definition}
\newtheorem{theorem}{定理}[section]
\newtheorem*{theorem*}{定理}
\newtheorem{definition}[theorem]{Definition}
\newtheorem{definition*}{Definition}
\newtheorem{example}[theorem]{Example}
\newtheorem*{example*}{Example}
\newtheorem{proposition}[theorem]{Proposition}
\newtheorem*{proposition*}{Proposition}
\newtheorem{Note}[theorem]{Note}
\newtheorem*{Note*}{Note}
\newtheorem{Ntc}[theorem]{Notice}
\newtheorem*{Ntc*}{Notice}
\newtheorem{lemma}[theorem]{Lemma}
\newtheorem*{lemma*}{Lemma}
\newtheorem{Fact}[theorem]{Fact}
\newtheorem*{Fact*}{Fact}
\newtheorem{question}[theorem]{Question}
\newtheorem*{question*}{Question}
\newtheorem{conjecture}[theorem]{予想}
\newtheorem*{conjecture*}{予想}
\newtheorem{Rule}[theorem]{Rule}
\newtheorem*{Rule*}{Rule}
\newtheorem{notation}[theorem]{Notation}
\newtheorem*{Not*}{Notation}
\newtheorem{corollary}[theorem]{Corollary}
\newtheorem*{corollary*}{Corollary}
\newtheorem{remark}[theorem]{Remark}
\newtheorem*{Rmk*}{Remark}
\newtheorem{condition}[theorem]{条件}
\newtheorem*{condition*}{条件}
\newtheorem{Conv}[theorem]{Convention}
\newtheorem*{Conv*}{Convention}
\newtheorem{observation}[theorem]{Observation}
\newtheorem*{observation*}{Observation}
%%%% newcommand

%%%logic symbol
\newcommand{\deq}{\coloneqq}

\newcommand{\dbraket}[1]{\hspace{-1.5pt}\braket{\hspace{-2.2pt}\braket{#1}\hspace{-2.2pt}}}

\newcommand{\textcmd}[1]{\texttt{\symbol{"5C}#1}}

%%special sets
\newcommand{\emp}{\varnothing}%emptyset
\newcommand{\C}{\mathbb{C}}%complex number
\newcommand{\Ha}{\mathbb{H}}%quaternion
\newcommand{\F}{\mathbb{F}}%field
\newcommand{\R}{\mathbb{R}}%real number
\newcommand{\Q}{\mathbb{Q}}%rational number
\newcommand{\Z}{\mathbb{Z}}%integer
\newcommand{\N}{\mathbb{N}_{0}}%natural number
\newcommand{\Pj}{\mathbb{P}}%bold p
\newcommand{\vep}{\varepsilon}%varepsilon

%%%%

\newcommand{\mb}[1]{\mathbb{#1}}%blackboard bold (for math mode)
\newcommand{\mcal}[1]{\mathcal{#1}}%

\newcommand{\opn}[1]{\operatorname{#1}}
\newcommand{\catn}[1]{\mathbf{#1}}

\newcommand{\abk}[1]{\langle {#1} \rangle}%angle bracket 
\newcommand{\Abk}[1]{\left \langle {#1} \right \rangle}%angle bracket (auto sizing)
\newcommand{\dabk}[1]{\langle\! \langle {#1}\rangle \! \rangle}%double angle bracket
\newcommand{\Dabk}[1]{\left \langle \! \left \langle {#1} \right \rangle \! \right \rangle}%double angle bracket
\newcommand{\Sbk}[1]{\left[ {#1} ]\right }% square bracket [] (auto sizing)
\newcommand{\Cbk}[1]{\left \{ {#1}\right \}}% curly bracket {} (auto sizing)
\newcommand{\dcbk}[1]{\{\!\!\{ {#1}\}\!\!\}} % double curly bracket {{}} 
\newcommand{\Dcbk}[1]{\left \{\!\! \left \{ {#1} \right\} \!\!\right \}} % double curly bracket {{}} (auto sizing)
\newcommand{\Paren}[1]{\left ( {#1} \right )}%parenthesis () (auto sizing)
\newcommand{\dparen}[1]{(\!({#1})\!)}%double parenthesis
\newcommand{\xto}[1]{\xrightarrow{#1}}
\newcommand{\xgets}[1]{\xleftarrow{#1}}
\newcommand{\hookto}{\hookrightarrow}


%%%% 

%%%% mathabx.sty (font) 
\DeclareFontFamily{U}{matha}{\hyphenchar\font45}
\DeclareFontShape{U}{matha}{m}{n}{
      <5> <6> <7> <8> <9> <10> gen * matha
      <10.95> matha10 <12> <14.4> <17.28> <20.74> <24.88> matha12
      }{}
\DeclareSymbolFont{matha}{U}{matha}{m}{n}

\DeclareFontFamily{U}{mathb}{\hyphenchar\font45}
\DeclareFontShape{U}{mathb}{m}{n}{
      <5> <6> <7> <8> <9> <10> gen * mathb
      <10.95> mathb10 <12> <14.4> <17.28> <20.74> <24.88> mathb12
      }{}
\DeclareSymbolFont{mathb}{U}{mathb}{m}{n}

\DeclareFontFamily{U}{mathx}{\hyphenchar\font45}
\DeclareFontShape{U}{mathx}{m}{n}{
      <5> <6> <7> <8> <9> <10>
      <10.95> <12> <14.4> <17.28> <20.74> <24.88>
      mathx10
      }{}
\DeclareSymbolFont{mathx}{U}{mathx}{m}{n}

%DeclareMathSymbol (from mathabx.sty)
\DeclareMathSymbol{\bigboxslash}{\mathop}{mathx}{"FE}
\DeclareMathSymbol{\bigboxtimes}{\mathop}{mathx}{"D2}
%%%%

%%%% MnSymbol.sty (font)
\DeclareFontFamily{U}{MnSymbolC}{}
\DeclareFontShape{U}{MnSymbolC}{m}{n}{
  <-6> MnSymbolC5
  <6-7> MnSymbolC6
  <7-8> MnSymbolC7
  <8-9> MnSymbolC8
  <9-10> MnSymbolC9
  <10-12> MnSymbolC10
  <12-> MnSymbolC12}{}
\DeclareFontShape{U}{MnSymbolC}{b}{n}{
  <-6> MnSymbolA-Bold5
  <6-7> MnSymbolC-Bold6
  <7-8> MnSymbolC-Bold7
  <8-9> MnSymbolC-Bold8
  <9-10> MnSymbolC-Bold9
  <10-12> MnSymbolC-Bold10
  <12-> MnSymbolC-Bold12}{}

\DeclareSymbolFont{MnSyC}{U}{MnSymbolC}{m}{n}

%%%% DeclareMathSymbol (from MnSymbol.sty)

\DeclareMathSymbol{\tplus}{\mathbin}{MnSyC}{43}
\DeclareMathSymbol{\aplus}{\mathbin}{MnSyC}{190}

%%%% renewcommand




%%%% footnote

\newcommand{\cfootnote}[1]{\footnote{#1}}

\newcommand{\myfootnote}[1]{\hspace{-5pt}\footnote{#1}}

\newcommand{\TB}{\mcal{T}_{B_0}}
\newcommand{\TBZ}{\mcal{T}_{\Z,B_0}}
\newcommand{\AffS}{{\mathop{\mcal{A}\!f\!\!f\!}\nolimits}}
\newcommand{\FBZ}{\mcal{F}_{\Z,B}}
\newcommand{\FB}{\mcal{F}_{B}}
\newcommand{\simto}{ 
\mathrel{\raisebox{0.13em}{${\sim}$}}
\kern -0.75em \mathrel{\raisebox{-0.11em}{${\scriptstyle \to}$}}  
}
%%%% from 

%%%% from  https://tex.stackexchange.com/questions/183702/formatting-back-references-in-bibliography-bibtex
\renewcommand*{\backrefalt}[4]{
    \ifcase #1 [Not cited.]%
        \or        [Cited on p.#2.]%
        \else      [Cited on p.#2.]%
    \fi}


\usepackage{mathrsfs}
\usepackage{upgreek}
\numberwithin{equation}{section}
\title{On counting lattice points in some tropical spaces and beyond
}
\author[Y. Tsutsui]{Yuki Tsutsui}
% \address{Graduate School of Mathematical Sciences,
% The University of Tokyo, 3-8-1 Komaba, Meguro-Ku,
% Tokyo, 153-8914, Japan}
% \email{tyuki@ms.u-tokyo.ac.jp}
% \date{\today}

\begin{document}

論文題目:

On graded modules associated with line bundles on 
tropical curves and integral Hessian manifolds
(トロピカル曲線と整ヘッセ多様体上の直線束に付随した
次数付き加群について)

氏名: 筒井 勇樹

本論文の目的は (1)トロピカル幾何学上の直線束を表現する
良い条件をもつ滑らかな因子データに対して Euler 数を定義すること
と(2)その定式化から自然と提起される Riemann--Roch 型の定理の
新たな類似をトロピカル直線と Hesse 計量をもつ整アフィン多様体に対して
証明することである。本論文の定式化におけるトロピカル曲線上の
直線束の Riemann--Roch 型の定理の類似は Gathmann--Kerber
によるトロピカル Riemann--Roch の定理とは異なるものである。

\section{Riemann--Rochの定理のトロピカル幾何的類似の研究の背景}

トロピカル幾何とは、ある種の凸多面体複体の台空間 $X$ とその上の
局所的に整数係数アフィンな関数のなす層 
$\mathcal{O}_X^{\times}$
の組を代数多様体の類似物とみなして研究する分野である
\cite{mikhalkinTropicalEigenwaveIntermediate2014a,
gross2019sheaftheoretic}。
そのような空間をトロピカル多様体と呼ぶ。
一次元トロピカル多様体をトロピカル曲線と呼び、
トロピカル多様体であり境界なし微分多様体の構造を自然にもつ空間を
整アフィン多様体と呼ぶ。
Hesse 計量は整アフィン多様体における
K\"ahler 計量の類似である。

コンパクトトロピカル曲線 $C$ 上の Cartier 因子 
$D\in \opn{CDiv}(C)$
に関する Riemann--Roch の定理は
Gathmann--Kerber
\cite{gathmannRiemannRochTheoremTropical2008a} によって 
Baker--Norine \cite{MR2355607}による
グラフの Riemann--Roch の定理の一般化として
次のような等式として証明された。
\begin{align} \label{equation-tropical-rr}
r(D)-r(K_C-D)=\opn{deg}(D)+\chi_{\mathrm{top}}(C)
\end{align}
ここで、$r(D)$ とは、$D$ の因子の線形系の階数であり、
$K_C$ は$C$ の標準因子、$\chi_{\mathrm{top}}(C)$ は
$C$ の位相的 Euler 数である。
この等式を高次元の場合に一般化することは自然な考えであるが
両辺ともに技術的な困難が存在する。代数幾何における
Hirzebruch--Riemann--Roch の定理では、
左辺は因子に付随する直線束の層コホモロジー群の
Euler 数に、右辺は 直線束の Chern 類と Todd 類の
カップ積で表現されるが、トロピカル幾何においては
直線束の層コホモロジー群と Todd 類の類似物は一般には
定義されていない。特に深刻なのは左辺であり、
トロピカル直線束はアーベル群のなす層ではないため、
層コホモロジー群を代数幾何の場合のように直接定義する
ことができない。一方で、数多くの研究結果から
トロピカル多様体の構造層の Euler 数の類似は
トロピカル多様体上の定数層の Euler 数であると
考えられている。

\section{SYZ予想および本研究の主結果}

本研究では、Strominger--Yau--Zaslow 予想 (SYZ予想)
のアイデアと超局所層理論\cite{MR1299726}を
利用することで、
境界を持たないトロピカル多様体上の直線束を表現する 
admissible な滑らかな因子に対して
Floer 複体もしくは Morse 複体の類似の
次数付き加群を定義し、その次数つき加群の 
Euler 数を 直線束の Euler 数のトロピカル類似とする。

SYZ 予想\cite{stromingerMirrorSymmetryTduality1996}
とは、Calabi--Yau 多様体 $\check{X}$ の
ミラー Calabi--Yau 多様体 $X$ が、
ある位相多様体 $B$ への(特異ファイバーを許す)
特殊 Lagrange トーラスのファイブレーション
$\check{f}_{B}\colon \check{X}\to B$ の
双対トーラスファイブレーション $f_{B}\colon X\to B$
として得られるという弦理論にかかわる数理物理学における
予想であり、この予想の数学的定式化および証明が 
Kontsevich のホモロジカルミラー対称性予想
\cite{MR1403918}の
証明を与えることが期待されている。

ホモロジー的ミラー対称性とは、
Calabi--Yau 多様体 $\check{X}$ に対してある
Calabi--Yau 多様体 $\mathcal{X}$ が存在し、
$\check{X}$ の導来深谷圏 $\catn{fuk}(\check{X})$ と
連接層の導来圏 $\catn{coh}(\mathcal{X})$ が
導来圏として擬同型
\begin{align} \label{equation-hms}
\catn{fuk}(\check{X})\simeq \catn{coh}(\mathcal{X})
\end{align}
であることを主張する予想である。
SYZ予想の数学的定式化は、擬同型関手
$\Phi_{\opn{SYZ}}\colon \catn{fuk}(\check{X})
\to \catn{coh}(\mathcal{X})$ の具体的な構成法を
与えると期待されており、
$\Phi_{\opn{SYZ}}$ を SYZ変換と呼ぶ。

ホモロジー的ミラー対称性 (\ref{equation-hms}) が
存在するならば、Floer コホモロジーと Ext 群との間に
\begin{align}
\opn{HF}^{\bullet}(\mathscr{L}_1,\mathscr{L}_2)
\simeq \opn{Ext}^{\bullet}_{\mathcal{O}_{\mathcal{X}}}(\Phi_{\opn{SYZ}}(\mathscr{L}_1),
\Phi_{\opn{SYZ}}(\mathscr{L}_2))
\end{align}
があることが導かれる。ここで
$\mathscr{L}_1$, $\mathscr{L}_2$ は、$\catn{fuk}(\check{X})$
の対象であり、コンパクト Lagrange 部分多様体といくつかの追加
データによって構成されている。$[L_1]$, $[L_2]$ をそれぞれ
$\mathscr{L}_1$, $\mathscr{L}_2$ の Lagrange 部分多様体
のホモロジー類だとすると、定義より
\begin{align}
\chi(\opn{HF}^{\bullet}(\mathscr{L}_1,\mathscr{L}_2))=
(-1)^{\frac{n(n-1)}{2}} \sharp ([L_1]\cap [L_2])
\end{align}
であることが導かれる。以上は Calabi--Yau 多様体に関する
ホモロジー的ミラー対称性予想についてだが、
$\mathcal{X}(B)$ をトーリック多様体に一般化したもの
\cite{MR2449059}や、
Mumford 曲線に一般化したもの\cite{auroux2022lagrangian}
も考察されている。その際の$\check{X}(B)$ 
はシンプレクティック的構造をStrataとして持つ様な空間に
置き換えられる。

$B$ をコンパクト整アフィン多様体とする。このとき、$B$ の
構造から誘導される特殊Lagrangeトーラス束 
$\check{X}(B)\to B$ と 
双対トーラス束 $\mathcal{X}(B)\to B$ が定義される。
$B$ 上のトロピカル直線束は、$H^{1}(B;\mathcal{O}^{\times}_B)$
によってパラメトライズされており、滑らかな関数のなす層 
$\mathcal{C}^{\infty}(B)$ への単射 
$\mathcal{O}_{B}^{\times}\to \mathcal{C}^{\infty}(B)$ が誘導
する $H^{0}(B;\mathcal{C}^{\infty}(B)/\mathcal{O}_{B}^{\times})$
から $H^{1}(B;\mathcal{O}^{\times}_B)$ への全射を持つ。
$H^{0}(B;\mathcal{C}^{\infty}(B)/\mathcal{O}_{B}^{\times})$
の元は、整アフィン多様体上のトロピカル Cartier 因子と
類似するものであり、本論文では区別のため滑らかな因子と
読んでいる。このとき、次の同型
\begin{align} \label{equation-lagrangian-section}
H^{0}(B;\mathcal{C}^{\infty}(B)/\mathcal{O}_{B}^{\times})
\simeq \{s\colon B\to \check{X}(B)\mid 
\check{f}_{B}\circ s=\opn{id}_B, \,
\opn{Im}(s) \text{ が Lagrange 部分多様体} \}
\end{align}
が成り立つ。\ref{equation-lagrangian-section} の
右辺に属する元を Lagrange 切断と呼び、
自然と $\catn{fuk}(\check{X}(B))$ の対象
$\mathscr{L}_s$ とすることができ、
SYZ 変換によって$\mathcal{X}(B)$ 上の直線束へと
変換される。同型 (\ref{equation-lagrangian-section})
より Lagrange 切断は $B$ の開被覆 $(U_i)_{i\in I}$ 
とその上の関数
Lagrange 切断 $s$ が、
零切断 $s_0$ と横断的に交わる時、
$p\in s_0\cap s$の開近傍上の関数$f_p\colon U_p\to \mathbb{R}$
で、$df_p=s|_{U_p}$ かつ $p$で孤立特異点を持つもので 
$\opn{HF}^{\bullet}(\mathscr{L}_{s_0},\mathscr{L}_s)$ を
定義する複体は次のようになる。
\begin{align}
\opn{CF}^{\bullet}(\mathscr{L}_{s_0},
\mathscr{L}_s)=\bigoplus_{p\in \check{f}_{B}(\opn{Im}(s_0)\cap \opn{Im}(s))}
\Lambda^{\mathbb{C}}_{\mathrm{nov}}[-\opn{ind}_{\mathrm{Morse}}(f_p,p)]
\end{align}
と表現することができる\cite{MR1882331}。ここで
$\opn{ind}_{\mathrm{Morse}}(f_p,p)$
とは、$f_p$ の $p$ での Morse指数であり、
$\Lambda^{\mathbb{C}}_{\mathrm{nov}}$ は
$\mathbb{C}$ 上の Novikov 体である。
ここで注目するべき点は、
$\opn{HF}^{\bullet}(\mathscr{L}_{s_0},\mathscr{L}_{s})$
は SYZ変換によって層コホモロジー
$H^{\bullet}(\mathcal{X}(B);
\Phi_{\opn{SYZ}}(\mathscr{L}_{s}))$
と同型であり\cite{MR4301560}、
$\opn{CF}^{\bullet}(\mathscr{L}_{s_0},
\mathscr{L}_s)$は次数つきベクトル空間としては、
トロピカル幾何の言葉のみで表現されていることの二点である。

これらを踏まえた上で博士論文提出者は、SYZ 予想のトイモデルとして
次の予想を提示する。

\begin{conjecture} \label{conjecture-mirror-tropical-rr}

$X$ をコンパクトなトロピカル多様体とし、
$\mathcal{L}$ を $X$ 上のトロピカル直線束とする。
ここで、$X$ には `Todd 類' $\opn{td}(X)$ が
トロピカルコホモロジー群 $H^{\bullet,*}(X;\mathbb{Z})$
の元として定義されていると仮定する。
このとき、$\mathcal{L}$ を表現するしかるべき条件を満たした
`滑らかな因子 $s$' が存在し、$s$ が定義する次数つき
$\mathbb{Z}$-加群 $\opn{LMD}^{\bullet}(X;s)$ に対して、
$s$ の取り方によらず
次の等式が成り立つ:
\begin{align}
\chi(\opn{LMD}^{\bullet}(X;s))=\int_{X}\opn{ch}(\mathcal{L})\opn{td}(X).
\end{align}
ここで、$\int_X$ はトレース写像であり、
$\mathrm{ch}(\mathcal{L})$ は、$\mathcal{L}$ の全 Chern 類である。
\end{conjecture}
$\opn{LMD}^{\bullet}(X;s)$ は次節で述べるが、
コンパクト整アフィン多様体 $B$と横断的に交わる Lagrange 切断$s$
に対しては、次数つきベクトル空間として
$\opn{LMD}^{\bullet}(B;s)\otimes_{\mathbb{Z}}
\Lambda^{\mathbb{C}}_{\mathrm{nov}}=
\opn{CF}^{\bullet}(\mathscr{L}_{s_0},
\mathscr{L}_s)$ となる。

上述の $X$ 上の滑らかな因子 $s$ のしかるべき条件とは、本論文では次のような条件である。

\begin{condition} \label{condition-good}
(i) $X$ の座標系がすべて $\mathbb{R}^{n}$ 
の凸多面体複体の開集合として取れるトロピカル多様体である。
(ii) $s$ が admissible である。
(iii) $\sharp (\opn{Im}(s_0)\cap \opn{Im}(s))<\infty$。
(iv) $\opn{LMD}^{\bullet}(X;s)$ が有限生成である。
\end{condition}
なお最後の条件は将来的に不要となると期待している。
いくつか用語の説明を後回しにして、
以上を踏まえた上での本論文の主結果(の弱い版)
を述べると以下のようになる。

\begin{theorem} \label{theorem-main}
$C$ (resp. $B$) をコンパクトなトロピカル曲線 (resp. Hesse 計量を持つ
整アフィン多様体) とし、$\mathcal{L}$ をその上の直線束
$s$ を $C$ (resp. $B$) 上の
\cref{condition-good} を満たす滑らかな因子 $s$ 
とする。このとき、次が成り立つ。
\begin{align}
\chi(\opn{LMD}^{\bullet}(C;s))=\int_C c_1(\mathcal{L})+
\chi_{\opn{top}}(C), \quad
\chi(\opn{LMD}^{\bullet}(B;s))=\int_B \frac{c_1(\mathcal{L})^n}{n!}
\end{align}
\end{theorem}
トロピカル曲線の定理\ref{theorem-main} の右辺は
(\ref{equation-tropical-rr}) の右辺と同じである。
一般にトロピカル多様体の Todd 類 は定義されていないが、
トロピカル多様体の Chern 類が
満たすべき条件から $\opn{td}(B)=1$ であるべきであることが
わかるため、定理\ref{theorem-main} は
\cref{conjecture-mirror-tropical-rr}
への肯定的な部分解答であるといえる。
この二つの結果はSYZ変換を介して \cite{MR4301560} や 
\cite{auroux2022lagrangian} の特殊例ともみなせるが、
本提出論文の証明はトロピカル幾何学の言葉のみで完結しており、
左辺に関しても上述のものとは異なるものである。
\section{局所 Morse データと証明の概略}
この節では、\cref{condition-good} (i) を満たしている
有理凸多面体空間 $S$ とその上の滑らかな因子
$s\in H^{0}(S;\mathcal{A}_{S}^{0,0}/\mathcal{O}^{\times}_S)$
に対する admissible 条件と 
$\opn{LMD}^{\bullet}(S;s)$ について説明する。
ここで $\mathcal{A}^{0,0}_S$ は $S$ 上の $(0,0)$-superform
のなす層のことである\cite{MR3903579}。まず、$s$は局所的には、$X$の開集合上の 
$(0,0)$ superform $f\colon U\to \mathbb{R}$
によって表現されており、各点 $x$ において
全微分ベクトル $df(x)\in(\mathcal{F}_{S}^{1})_x$ 
が定義される。$\mathcal{F}_{S}^{1}$ とは、定数層
$\mathbb{R}_S$ による
商層 $\mathcal{O}^{\times}_{S}/\mathbb{R}_{S}$ 
の $\mathbb{R}$ 係数拡大であり、
$(\mathcal{F}_{S}^{1})_x$の双対空間
$T_x S$は、$S$ の $x$ の局所的な錐
$\opn{LC}_x S$ の張るベクトル空間と同一視される。
$\phi \colon \opn{LC}_x S\to T_x S$ を上述の同一視の下
による閉埋め込みとすると
$\opn{SS}_x (S)$を
$\opn{SS}_0(\phi_* \phi^{-1}\mathbb{Z}_{T_x S})$
と定義する。このとき、
$\opn{SS}_0(\phi_* \phi^{-1}\mathbb{Z}_{T_x S})$は、
$T_x X$を微分多様体とみなしたときの、
$\phi_* \phi^{-1}\mathbb{Z}_{T_x S}$ のマイクロ台と
原点での$T_x X$の余接ベクトル空間との交叉である。
このとき、
$f\colon U \to \mathbb{R}$ が $x$ にて admissible 
とは、次の条件を満たすことである。
\begin{align}
df(x)\notin\opn{span}_{\mathbb{R}}(\opn{SS}_x(S))+
(\mathcal{F}_{\mathbb{Z},S}^{1})_x 
\setminus (\mathcal{F}_{\mathbb{Z},S}^{1})_x
\end{align}

$S$ には、各点に対してトーラス
$\check{X}_0(S)_x\deq ((\mathcal{F}^{1}_{\mathbb{Z},S})_x/
\opn{span}_{\mathbb{R}}(\opn{SS}_x(S))\cap 
(\mathcal{F}^{1}_{\mathbb{Z},S})_x)\otimes_{\mathbb{Z}}
\mathbb{R}/\mathbb{Z}$ が定義され、
admissible な滑らかな因子 $s$ に対して写像
$\check{s}\colon S\to \check{X}_0(S)\deq 
\bigcup_{x\in S} \check{X}_0(S)_x$ が定義される。 
そして標準的な射影 $\check{f}_{S}\colon \check{X}_0(S)\to S$
によって、自明な因子 $s_0$ と $s$ の交点は
$s_0\cap s\deq \check{f}_{S}(\opn{Im}(\check{s}_0)\cap 
\opn{Im}(\check{s}))$ と定義される。
この、$s_0\cap s$ が有限集合のとき、
$s$ の局所 Morse データは層の局所 Morse データを用いることで
次のように定義される。
\begin{align} \label{equation-local-morse-data}
\opn{LMD}^{\bullet}(S;s)\deq 
\bigoplus_{p\in s_0\cap s} 
(R^{\bullet}_{\{x\in U_p\mid f_p(x)\geq f_p(p)\}}(\mathbb{Z}_{U_p}))_p.
\end{align}
ここで、$f_p\colon U_p\to \mathbb{R}$ とは、
$p$ の十分小さな近傍 $U_p$ 上の $(0,0)$-superform で
$s|_{U_p}$ が $f_p$ の定義する滑らかな主因子と一致し、
$p$ にのみ特異点を持つものである。 
\cref{equation-local-morse-data}の右辺の直和成分は
局所 Morse データもしくは、micro-stalk と呼ばれるものの
コホモロジー群であり、超局所層理論において基本的なものである
\cite{MR2031639,MR1299726}。

コンパクトトロピカル曲線における、定理\ref{theorem-main}の証明は、
各閉区間に対して定理\ref{theorem-main}を証明し、
それらを張り合わせることで証明する。
Hesse 計量をもつコンパクト整アフィン多様体については、
Hesse 計量をもつコンパクト整アフィン多様体は Cheng--Yau
の結果\cite{MR714338}より、トロピカルトーラスの有限不分岐被覆によって
得られることとトロピカルコホモロジーとトロピカル Borel--Moore
ホモロジーとのカップ積に関する射影公式
\cite{gross2019sheaftheoretic}を利用することで
トロピカルトーラスの場合に帰着させることで示した。
また本論文では、有理多面体空間 $S$ 上の滑らかな因子 $s$ と
有理多面体空間 $S'$上の滑らかな因子$s'$が\cref{condition-good}
を満たしかつコホモロジー的な Milnor 条件を満たすとき、
K\"unneth 型の定理が成り立つことを層における
Thom--Sebastiani型の定理\cite{MR2031639}を
用いることで示している。

\bibliography{lattice_points_surface}
\bibliographystyle{halpha}

\end{document}