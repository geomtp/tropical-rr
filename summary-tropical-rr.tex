\documentclass[uplatex,dvipdfmx,12pt]{jsarticle}

%%%% packages
\usepackage[utf8]{inputenc}
\usepackage[dvipdfmx]{graphicx,color}%for images
\usepackage{bm}%fonts
\usepackage{tikz-cd}%
\usetikzlibrary{cd}
\usetikzlibrary{calc}
\usepackage{amsmath,amsthm,amstext,amsfonts,amsbsy}% ほぼ必須
\usepackage{amssymb}
\usepackage{latexsym}% ほぼ必須
\usepackage{makecell}%表のセル内で改行するためのパッケージ
\usepackage{algpseudocode,algorithm}%疑似コード用
%\usepackage{todonotes}%comments
\usepackage[top=0.8in,left=0.8in,right=0.8in,
bottom=1.1in]{geometry}
\usepackage{layout}
\usepackage[T1]{fontenc}%font encoding
\usepackage{physics}
\usepackage{braket}%after physics
\usepackage{mathtools,thmtools}
\usepackage{imakeidx}%before hyperref for pagebackref
\usepackage[dvipdfmx]{hyperref}
\usepackage[capitalize]{cleveref}
\hypersetup{
     colorlinks = true,
     citecolor  = blue,
     linkcolor  = blue, 
     urlcolor   = blue, 
}
\usepackage{mathrsfs}
\usepackage{upgreek}
\numberwithin{equation}{section}

\usepackage{titlesec}

\titleformat*{\section}{\large\bfseries}

%%%% theoremstyle

\theoremstyle{definition}
\newtheorem{theorem}{Theorem}[section]
\newtheorem*{theorem*}{定理}
\newtheorem{definition}[theorem]{Definition}
\newtheorem{definition*}{Definition}
\newtheorem{example}[theorem]{Example}
\newtheorem*{example*}{Example}
\newtheorem{proposition}[theorem]{Proposition}
\newtheorem*{proposition*}{Proposition}
\newtheorem{Note}[theorem]{Note}
\newtheorem*{Note*}{Note}
\newtheorem{Ntc}[theorem]{Notice}
\newtheorem*{Ntc*}{Notice}
\newtheorem{lemma}[theorem]{Lemma}
\newtheorem*{lemma*}{Lemma}
\newtheorem{Fact}[theorem]{Fact}
\newtheorem*{Fact*}{Fact}
\newtheorem{question}[theorem]{Question}
\newtheorem*{question*}{Question}
\newtheorem{conjecture}[theorem]{Conjecture}
\newtheorem*{conjecture*}{予想}
\newtheorem{Rule}[theorem]{Rule}
\newtheorem*{Rule*}{Rule}
\newtheorem{notation}[theorem]{Notation}
\newtheorem*{Not*}{Notation}
\newtheorem{corollary}[theorem]{Corollary}
\newtheorem*{corollary*}{Corollary}
\newtheorem{remark}[theorem]{Remark}
\newtheorem*{Rmk*}{Remark}
\newtheorem{condition}[theorem]{Condition}
\newtheorem*{condition*}{条件}
\newtheorem{Conv}[theorem]{Convention}
\newtheorem*{Conv*}{Convention}
\newtheorem{observation}[theorem]{Observation}
\newtheorem*{observation*}{Observation}
\newtheorem{expectation}[theorem]{メタ予想}
\newtheorem{fact}[theorem]{事実}

% newcommand
\newcommand{\deq}{\coloneqq}
\newcommand{\opn}[1]{\operatorname{#1}}
\newcommand{\catn}[1]{\mathbf{#1}}
\newcommand{\beforesection}{\vspace{-15pt}}
\newcommand{\aftersection}{\vspace{-8pt}}

\DeclareMathOperator{\tform}{\Omega}

\title{論文内容の要旨
}
%\author[Y. Tsutsui]{Yuki Tsutsui}
% \address{Graduate School of Mathematical Sciences,
% The University of Tokyo, 3-8-1 Komaba, Meguro-Ku,
% Tokyo, 153-8914, Japan}
% \email{tyuki@ms.u-tokyo.ac.jp}
\date{}

%layout

%package

\usepackage{lineno}

\newcommand*\patchAmsMathEnvironmentForLineno[1]{
  \expandafter\let\csname old#1\expandafter\endcsname\csname #1\endcsname
  \expandafter\let\csname oldend#1\expandafter\endcsname\csname end#1\endcsname
  \renewenvironment{#1}
     {\linenomath\csname old#1\endcsname}
     {\csname oldend#1\endcsname\endlinenomath}}
\newcommand*\patchBothAmsMathEnvironmentsForLineno[1]{
  \patchAmsMathEnvironmentForLineno{#1}
  \patchAmsMathEnvironmentForLineno{#1*}}
\AtBeginDocument{
\patchBothAmsMathEnvironmentsForLineno{equation}
\patchBothAmsMathEnvironmentsForLineno{align}
\patchBothAmsMathEnvironmentsForLineno{flalign}
\patchBothAmsMathEnvironmentsForLineno{alignat}
\patchBothAmsMathEnvironmentsForLineno{gather}
\patchBothAmsMathEnvironmentsForLineno{multline}
}

\usepackage{framed}
\setlength{\OuterFrameSep}{-4pt}
\setlength{\FrameSep}{3pt}

\begin{document}
\setlength{\baselineskip}{-5pt}
\setlength{\parskip}{2pt}

%\linenumbers

\maketitle
{\large
\noindent
論文題目:

Graded modules associated 
with permissible $C^{\infty}$-divisors on tropical manifolds

(トロピカル多様体上の可容$C^{\infty}$因子に付随した次数付き加群)
}

\vspace{5pt}

\noindent
{\large
氏名: 筒井 勇樹
}

\vspace{10pt}

\setlength{\baselineskip}{4pt}
\setlength{\parskip}{3pt}

本論文の目的は
(1)コンパクトなトロピカル多様体上の直線束を表現する
可容$C^{\infty}$因子に対して次数付き加群を定義すること
と
(2)その定式化が示唆するRiemann--Roch型の定理の
新たなトロピカル類似を
トロピカル曲線と
Hesse計量を許容する整アフィン多様体に対して
証明することである.
なお,このトロピカル類似は
Gathmann--Kerberによるトロピカル Riemann--Roch の定理とは異なる.

\beforesection

\section{Riemann--Rochの定理のトロピカル幾何的類似の研究の背景}

\aftersection

$\underline{\mathbb{R}}^{n} \deq (\mathbb{R}\cup\{-\infty\})^{n}$内の
有理多面集合$S$と
その上の整数係数の傾きを持つアフィン関数のなす層 
$\mathcal{O}_S^{\times}$
からなる組
$(S, \mathcal{O}_S^{\times})$
と局所的に同型であるような空間を
有理多面空間(rational polyhedral space)
と呼び,
その中のある特別なクラスをトロピカル多様体と呼ぶ.
トロピカル多様体を代数多様体の類似物とみなして研究する分野が
トロピカル幾何学である
\cite{mikhalkinTropicalEigenwaveIntermediate2014a,
gross2019sheaftheoretic}.
次数$1$の頂点を持たない有限距離グラフは標準的な
トロピカル曲線(一次元トロピカル多様体)の構造を持つ.
また,
可微分多様体の構造を持つトロピカル多様体は,
整アフィン多様体(つまり,変換関数が整アファイン関数であるような
アトラスを持つ可微分多様体)と自然に対応する.
Hesse 計量は整アフィン多様体における
K\"ahler 計量の類似である.

コンパクトなトロピカル曲線 $C$ 上の
因子$D$に対する
Riemann--Roch の定理は,
Baker--Norine による
グラフの Riemann--Roch の定理の一般化として,
Gathmann--Kerber
\cite{gathmannRiemannRochTheoremTropical2008a} によって 
次のような等式として証明された.
\begin{align} \label{equation-tropical-rr}
r(D)-r(K_C-D)=\opn{deg}(D)+\chi_{\mathrm{top}}(C)
\end{align}
ここで$r(D)$は因子$D$の線形系の階数であり,
$K_C$ は$C$ の標準因子,$\chi_{\mathrm{top}}(C)$ は
$C$ の位相的 Euler 数である.
この等式を高次元に一般化することは
自然な問題であるが,
右辺と左辺のどちらも高次元化は容易ではない.
特に左辺の高次元化には,
トロピカル直線束がAbel群の層ではないことから来る
本質的な困難がある.

\beforesection

\section{可容条件とコホモロジカル局所 Morse データ}

\aftersection

本論文では,Strominger--Yau--Zaslow 予想
\cite{stromingerMirrorSymmetryTduality1996}
に基づくアイデアと超局所層理論\cite{MR1299726}を
用いることで,
コンパクトなトロピカル多様体$S$上の直線束を表現する 
可容$C^{\infty}$因子$s$に対して
次数付き加群$\opn{LMD}^{\bullet}(S;s)$を定義し,
そのEuler数を
代数多様体上の直線束のEuler数のトロピカル類似として採用する.

まず,$S$の各点$x$には,余接ベクトル空間
$(\tform^{1}_{S})_x\deq
(\mathcal{O}_S^{\times}/\mathbb{R}_S)_x
\otimes_{\mathbb{Z}}\mathbb{R}$
と,その中の整数ベクトルのなす格子
$(\tform^{1}_{\mathbb{Z},S})_x\deq
(\mathcal{O}_S^{\times}/\mathbb{R}_S)_x$が存在する.
接ベクトル空間
$T_x S\deq(\tform^{1}_{S})_x^{\vee}$の中には,
local coneと呼ばれる
$T_xS$を張る有理多面集合
$\opn{LC}_x S$が存在する.
$\phi \colon \opn{LC}_x S\to T_x S$を埋め込み写像とすると,
$T^{*}(T_xS)$の部分集合である層のマイクロ台
$\opn{SS}(\phi_!\phi^{-1}\mathbb{Z}_{T_xS})$が定義される.
このとき,原点での余接ベクトル空間$T^{*}_0(T_xS)$が
$(\tform_S^{1})_x$と自然に同型であることに注意することで,
$(\tform_S^{1})_x$の部分集合
$\opn{SS}(S)_x\deq \opn{SS}(\phi_!\phi^{-1}\mathbb{Z}_{T_xS})
\cap (\tform_S^{1})_x$が定義される.

$S$上の$C^{\infty}$因子$s$は,$S$上の
weakly-smoothな関数のなす層$\mathcal{A}^{\mathrm{weak}}_S$
によって定義されるコホモロジー
$H^{0}(S;\mathcal{A}^{\mathrm{weak}}_S/\mathcal{O}^{\times}_S)$
の元である.
ここで,
有理多面空間$S$上の連続関数
$f\colon S\to \mathbb{R}$
がweakly-smoothであるとは,
各点$x\in S$に対して$x$を含むあるチャート
$\psi \colon U_x\to \underline{\mathbb{R}}^{n}$と,
$\mathbb{R}^{n}$の開集合上の
正値$C^{\infty}$関数$h\colon V\to \mathbb{R}_{>0}$
が存在して,
$f|_{U_x}=\opn{log}\circ h\circ \opn{exp}\circ \psi$
となることを指す
(Definition 2.44とDefinition 2.45).
関数
$
f\colon \underline{\mathbb{R}}\to \mathbb{R}
$;
$
x\mapsto \opn{log}(1+e^{x})
$
はweakly-smoothな関数の典型例であり,
この時の
$
h \colon \mathbb{R}_{>-1} \to \mathbb{R}_{>0}
$
は
$
z \mapsto 1+z
$
で与えられる.
なお,$S$がboundarylessのとき,つまり
$S$ の座標系がすべてある$\mathbb{R}^{n}$ 
に含まれる有理多面集合の開集合として取れるとき,
$\mathcal{A}^{\mathrm{weak}}_S$は$S$上の
\cite{MR3903579}の意味の
(0,0)-superformのなす
層$\mathcal{A}^{0,0}_S$
と一致する.
上述の例は,
weakly-smoothだが$(0,0)$-superformではない
典型的な例を与える.

$C^{\infty}$因子$s$は局所的には
$S$の開集合上の 
weakly-smoothな関数 $f\colon U\to \mathbb{R}$
の主因子として表現されており,各点 $x$ において
全微分ベクトル $df(x)\in(\tform_{S}^{1})_x$ 
が定義される.
weakly-smoothな関数
$f\colon U \to \mathbb{R}$ が $x$ で
前可容(prepermissible)であるとは,
次の条件(Condition 2.26)を満たすことを指す.
\begin{align}
df(x)\notin\opn{span}_{\mathbb{R}}(\opn{SS}(S)_x)+
(\tform_{\mathbb{Z},S}^{1})_x 
\setminus (\tform_{\mathbb{Z},S}^{1})_x
\end{align}
$C^{\infty}$因子$s$が前可容であるとは,
$s$が前可容な関数からなる
局所データ$\{(U_i,f_i)\}_{i\in I}$
を持つことをいう.

$S$上の各点$x$に対してトーラス
\begin{equation}
\check{X}_0(S)_x\deq ((\tform^{1}_{\mathbb{Z},S})_x/
\opn{span}_{\mathbb{R}}(\opn{SS}_x(S))\cap 
(\tform^{1}_{\mathbb{Z},S})_x)\otimes_{\mathbb{Z}}
\mathbb{R}/\mathbb{Z}
\end{equation}
が定義され,
前可容な$C^{\infty}$因子 $s$ に対して写像
$\check{s}\colon S\to \check{X}_0(S)\deq 
\bigcup_{x\in S} \check{X}_0(S)_x$ が定義される. 
標準的な射影 $\check{f}_{S}\colon \check{X}_0(S)\to S$
によって,自明な因子 $s_0$ と $s$ の交点は
$s_0\cap s\deq \check{f}_{S}(\opn{Im}(\check{s}_0)\cap 
\opn{Im}(\check{s}))$ と定義できる
(Definition 2.29とProposition 2.34).
この$s_0\cap s$ が有限集合のとき,
$s$ の(コホモロジー的な)局所 Morse データは次のように定義される.
\begin{align} \label{equation-local-morse-data}
\opn{LMD}^{\bullet}(S;s)\deq 
\bigoplus_{p\in s_0\cap s} 
H^{\bullet}(R\Gamma_{\{x\in U_p\mid f_p(x)\geq f_p(p)\}}(\mathbb{Z}_{U_p})_p)
\end{align}
ここで,
$f_p\colon U_p\to \mathbb{R}$は
$p$ の十分小さな近傍 $U_p$ 上の weakly-smoothな関数で,
$s|_{U_p}$ が $f_p$ の定義する$C^{\infty}$因子と一致し,
$p$ にのみ臨界点を持つようなものである.

$C^{\infty}$因子$s$が
可容である(permissible)とは,
次の条件を満たす事を指す.

\setcounter{section}{2}
\setcounter{condition}{38}
\begin{condition} \label{condition-good}
(i) $s$ が前可容である.
(ii) $\sharp (s_0\cap s)<\infty$.
(iii) $\opn{LMD}^{\bullet}(S;s)$ が有限生成.
\end{condition}

$B$を向き付け可能なコンパクト整アフィン多様体とすると,
$\mathcal{A}_B^{\mathrm{weak}}$は$B$上の
$C^{\infty}$関数のなす層であり,
$B$上の$C^{\infty}$因子$s$は,
$B$上の自然なLagrangeトーラス束
$\check{f}_B\colon \check{X}(B)\to B$
のLagrange切断$L_s$に自然に対応する.
$\{(U_i,f_i)\}_{i\in I}$を$s$の局所データとする.
各$f_i$がMorse関数のとき,$L_s$は,
$\check{f}_B$の零切断$L_{s_0}$と横断的に交わる.
$B$が$\pi_2(B) = 0$を満たす時,
ホモロジー的ミラー対称性\cite{MR1882331,MR4301560}によって,
$L_{s}$はミラー多様体上の直線束に移り,
$\chi(\opn{LMD}^{\bullet}(S;s))$は
そのEuler数と一致する.

\section{主予想と主結果}

$S$を有理多面空間とすると,
次の短完全列が存在する.
\begin{align}
0\to \mathcal{O}_S^{\times} \to 
\mathcal{A}^{\mathrm{weak}}_S \to 
\mathcal{A}^{\mathrm{weak}}_S/\mathcal{O}_S^{\times}
\to 0
\end{align}
この短完全列は,
接続準同型
$
[\cdot] \colon 
H^{0}(S;\mathcal{A}^{\mathrm{weak}}_S/\mathcal{O}_S^{\times})
\to H^{1}(S;\mathcal{O}_S^{\times})
$;
$
s\mapsto [s]
$
を誘導する.
任意の元$\mathcal{L}\in H^{1}(S;\mathcal{O}^{\times}_S)$
に対して,
$[s]=\mathcal{L}$となる可容$C^{\infty}$因子
$s$が
存在するか否かは一般には非自明であるが,
豊富に存在すると予想している(Conjecture 1.1).
実際に,
$S$がコンパクトトロピカル曲線もしくは整アフィン多様体の場合は豊富に存在する.
同様に,
短完全列
\begin{equation}
0
\to \mathbb{R}_S
\to \mathcal{O}^{\times}_S
\to \tform_{\mathbb{Z},S}^{1}
\to 0
\end{equation}
が
準同型
$
c_1\colon H^{1}(S;\mathcal{O}^{\times}_S)
\to H^{1}(S;\tform_{\mathbb{Z},S}^{1})
$
を誘導する.
$\Omega_{\mathbb{Z},S}^{p}$を$S$のトロピカル$p$形式のなす層
とする\cite{gross2019sheaftheoretic}.
$S$がトロピカル多様体のとき,
$H^{p,q}(S;\mathbb{Z})\deq H^{q}(S;\tform_{\mathbb{Z},S}^{p})$は
\cite{mikhalkinTropicalEigenwaveIntermediate2014a}
で定義されたトロピカルコホモロジーと一致する.
全コホモロジー$H^{\bullet,\bullet}(S;\mathbb{Z})$
には自然な環構造が存在し,
これを用いて
$\mathcal{L}\in H^{1}(S;\mathcal{O}_S^{\times})$の
Chern指標を$\opn{ch}(\mathcal{L})\deq
\sum_{i=0}^{\infty}\frac{c_1(\mathcal{L})^{i}}{i!}$と定義することが
できる.
$S$が連結でコンパクトな
$n$次元トロピカル多様体であるとき,トレース写像
$\int_{S}\colon H^{n,n}(S;\mathbb{Z})\to \mathbb{Z}$が存在し,
$H^{\bullet,\bullet}(S;\mathbb{Z})$上へと自然に零延長される.
非連結な場合でも同様に定義される.
\setcounter{section}{1}
\setcounter{condition}{1}
\begin{conjecture} \label{conjecture-mirror-tropical-rr}
任意のコンパクトなトロピカル多様体$S$に対して,
$H^{\bullet,\bullet}(S;\mathbb{Z})$
の元$\opn{td}(S)$が存在して,
$S$上の任意の可容
$C^{\infty}$因子$s$に対し,
次が成り立つ.
\begin{align}
\chi(\opn{LMD}^{\bullet}(S;s))=
\int_{S}\opn{ch}([s])\opn{td}(S)
\end{align}
\end{conjecture}


本論文の主結果は,
特別な場合の
\cref{conjecture-mirror-tropical-rr}
に対する
肯定的な解答である.

\begin{theorem} \label{theorem-main-1}
$C$をコンパクトなトロピカル曲線とし,
$s$ を $C$上の可容$C^{\infty}$因子 
とする.このとき,次が成り立つ.
\begin{align}
\chi(\opn{LMD}^{\bullet}(C;s))=
\int_C c_1([s])+\frac{1}{2}c_1(-K_C)
=\opn{deg}([s])+\chi_{\opn{top}}(C). 
\end{align}
\end{theorem}

\begin{theorem}
\label{theorem-main-2}
$B$をHesse計量を許容するコンパクトな$n$次元整アフィン多様体とし,
$s$を可容$C^{\infty}$因子とする.このとき,次が成り立つ.
\begin{align}
\chi(\opn{LMD}^{\bullet}(B;s))=\int_B \frac{c_1([s])^n}{n!}
\end{align}
\end{theorem}

Theorem \ref{theorem-main-1}の証明には,
\cite{auroux2022lagrangian}で定義されているLagrange切断の
回転数の
可容$C^{\infty}$因子に対する類似を用いる.
また,Thereom \ref{theorem-main-1}の証明には,
Cheng--Yau \cite{MR714338}の結果から
Hesse計量を許容するコンパクトな整アファイン多様体が
トロピカルトーラスを有限不分岐被覆に持つことと,
トロピカル Borel--Moore
ホモロジーの射影公式
\cite{gross2019sheaftheoretic}を用いる.

\aftersection

{\small
\bibliography{lattice_points_surface}
\bibliographystyle{halpha}
}
\end{document}